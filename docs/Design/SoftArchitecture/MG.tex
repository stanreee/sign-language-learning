\documentclass[12pt, titlepage]{article}

\usepackage{fullpage}
\usepackage[round]{natbib}
\usepackage{multirow}
\usepackage{booktabs}
\usepackage{tabularx}
\usepackage{graphicx}
\usepackage{float}
\usepackage{hyperref}
\hypersetup{
    colorlinks,
    citecolor=blue,
    filecolor=black,
    linkcolor=red,
    urlcolor=blue
}

\input{../../Comments}
%% Common Parts

\newcommand{\progname}{Software Engineering} % PUT YOUR PROGRAM NAME HERE
\newcommand{\authname}{Team 15, ASLingo
\\ Andrew Kil
\\ Cassidy Baldin
\\ Edward Zhuang
\\ Jeremy Langner
\\ Stanley Chan} % AUTHOR NAMES                  

\usepackage{hyperref}
    \hypersetup{colorlinks=true, linkcolor=blue, citecolor=blue, filecolor=blue,
                urlcolor=blue, unicode=false}
    \urlstyle{same}
                                


\newcounter{acnum}
\newcommand{\actheacnum}{AC\theacnum}
\newcommand{\acref}[1]{AC\ref{#1}}

\newcounter{ucnum}
\newcommand{\uctheucnum}{UC\theucnum}
\newcommand{\uref}[1]{UC\ref{#1}}

\newcounter{mnum}
\newcommand{\mthemnum}{M\themnum}
\newcommand{\mref}[1]{M\ref{#1}}

\begin{document}

\title{Module Guide for \progname{}} 
\author{\authname}
\date{\today}

\maketitle

\pagenumbering{roman}

\section{Revision History}

\begin{tabularx}{\textwidth}{p{3cm}p{2cm}X}
\toprule {\bf Date} & {\bf Version} & {\bf Notes}\\
\midrule
Jan. 9, 2024 & 1.0 & Andrew and Edward; added module hierarchy\\
Jan. 10, 2024 & 1.1 & Stanley; added some module decompositions for hardware hiding and behaviour hiding modules\\
Jan. 10, 2024 & 1.2 & Andrew; rearranged Module Hierarchy and added decomposition descriptions for controller, hand sign recognition and verification modules\\
Jan. 11, 2024 & 1.3 & Jeremy and Cassidy added front end anticipated changes, modules and traceability mapping to these modules\\
\bottomrule
\end{tabularx}

\newpage

\section{Reference Material}

This section records information for easy reference.

\subsection{Abbreviations and Acronyms}

\renewcommand{\arraystretch}{1.2}
\begin{tabular}{l l} 
  \toprule		
  \textbf{symbol} & \textbf{description}\\
  \midrule 
  AC & Anticipated Change\\
  DAG & Directed Acyclic Graph \\
  M & Module \\
  MG & Module Guide \\
  OS & Operating System \\
  R & Requirement\\
  SC & Scientific Computing \\
  SRS & Software Requirements Specification\\
  \progname & Explanation of program name\\
  UC & Unlikely Change \\
  \wss{etc.} & \wss{...}\\
  \bottomrule
\end{tabular}\\

\newpage

\tableofcontents

\listoftables

\listoffigures

\newpage

\pagenumbering{arabic}

\section{Introduction}

Decomposing a system into modules is a commonly accepted approach to developing
software.  A module is a work assignment for a programmer or programming
team~\citep{ParnasEtAl1984}.  We advocate a decomposition
based on the principle of information hiding~\citep{Parnas1972a}.  This
principle supports design for change, because the ``secrets'' that each module
hides represent likely future changes.  Design for change is valuable in SC,
where modifications are frequent, especially during initial development as the
solution space is explored.  

Our design follows the rules layed out by \citet{ParnasEtAl1984}, as follows:
\begin{itemize}
\item System details that are likely to change independently should be the
  secrets of separate modules.
\item Each data structure is implemented in only one module.
\item Any other program that requires information stored in a module's data
  structures must obtain it by calling access programs belonging to that module.
\end{itemize}

After completing the first stage of the design, the Software Requirements
Specification (SRS), the Module Guide (MG) is developed~\citep{ParnasEtAl1984}. The MG
specifies the modular structure of the system and is intended to allow both
designers and maintainers to easily identify the parts of the software.  The
potential readers of this document are as follows:

\begin{itemize}
\item New project members: This document can be a guide for a new project member
  to easily understand the overall structure and quickly find the
  relevant modules they are searching for.
\item Maintainers: The hierarchical structure of the module guide improves the
  maintainers' understanding when they need to make changes to the system. It is
  important for a maintainer to update the relevant sections of the document
  after changes have been made.
\item Designers: Once the module guide has been written, it can be used to
  check for consistency, feasibility, and flexibility. Designers can verify the
  system in various ways, such as consistency among modules, feasibility of the
  decomposition, and flexibility of the design.
\end{itemize}

The rest of the document is organized as follows. Section
\ref{SecChange} lists the anticipated and unlikely changes of the software
requirements. Section \ref{SecMH} summarizes the module decomposition that
was constructed according to the likely changes. Section \ref{SecConnection}
specifies the connections between the software requirements and the
modules. Section \ref{SecMD} gives a detailed description of the
modules. Section \ref{SecTM} includes two traceability matrices. One checks
the completeness of the design against the requirements provided in the SRS. The
other shows the relation between anticipated changes and the modules. Section
\ref{SecUse} describes the use relation between modules.

\section{Anticipated and Unlikely Changes} \label{SecChange}

This section lists possible changes to the system. According to the likeliness
of the change, the possible changes are classified into two
categories. Anticipated changes are listed in Section \ref{SecAchange}, and
unlikely changes are listed in Section \ref{SecUchange}.

\subsection{Anticipated Changes} \label{SecAchange}

Anticipated changes are the source of the information that is to be hidden
inside the modules. Ideally, changing one of the anticipated changes will only
require changing the one module that hides the associated decision. The approach
adapted here is called design for
change.

\begin{description}
\item[\refstepcounter{acnum} \actheacnum \label{acInfo}:] The ASL vocabulary included in the program may be updated in the future as the language evolves.
\item[\refstepcounter{acnum} \actheacnum \label{acAbout}:] The About/Info Module may change if the developers want to change the formatting of the informational pages on the website (for example the home/resources page that use this module). 
\item[\refstepcounter{acnum} \actheacnum \label{acExercise}:] Exercise and Exercise Selection/History module may change if developers want to add in additional testing methods or material.
\item[\refstepcounter{acnum} \actheacnum \label{acAccount}:] Account management module may change if developers change what information is stored about the users progress. 
\end{description}

\subsection{Unlikely Changes} \label{SecUchange}

The module design should be as general as possible. However, a general system is
more complex. Sometimes this complexity is not necessary. Fixing some design
decisions at the system architecture stage can simplify the software design. If
these decision should later need to be changed, then many parts of the design
will potentially need to be modified. Hence, it is not intended that these
decisions will be changed.

\begin{description}
\item[\refstepcounter{ucnum} \uctheucnum \label{ucIO}:] Input/Output devices
  (Input: File and/or Keyboard, Output: File, Memory, and/or Screen).
\item[\refstepcounter{ucnum} \uctheucnum \label{ucLogin}:] The Login/Sign Up module is unlikely to change as we will be using user authentication methods and standard login procedures that are unlikely to change.
\end{description}

\section{Module Hierarchy} \label{SecMH}

This section provides an overview of the module design. Modules are summarized
in a hierarchy decomposed by secrets in Table \ref{TblMH}. The modules listed
below, which are leaves in the hierarchy tree, are the modules that will
actually be implemented.

\begin{description}
\item [\refstepcounter{mnum} \mthemnum \label{m1}:] Hand Sign Recognition Module
\item [\refstepcounter{mnum} \mthemnum \label{m2}:] Hand Sign Verification Module
% Need a middle-man to help frontend-backend communicate
% Ex: Front end changes the question, need to tell backend we're looking for a different sign now
\item [\refstepcounter{mnum} \mthemnum \label{m3}:] Controller Module
% Parsing raw data to processable data
\item [\refstepcounter{mnum} \mthemnum \label{m4}:] Data Collection Module
\item [\refstepcounter{mnum} \mthemnum \label{m5}:] Data Processing Module
\item [\refstepcounter{mnum} \mthemnum \label{m6}:] Machine Learning Module
\item [\refstepcounter{mnum} \mthemnum \label{m7}:] Testing and Verification Module
\item [\refstepcounter{mnum} \mthemnum \label{m8}:] Video Input Module
\item [\refstepcounter{mnum} \mthemnum \label{m9}:] About/Info Module
\item [\refstepcounter{mnum} \mthemnum \label{m10}:] Exercise Module
\item [\refstepcounter{mnum} \mthemnum \label{m11}:] Exercise Selection/History Module
\item [\refstepcounter{mnum} \mthemnum \label{m12}:] Login/Sign Up Module
\item [\refstepcounter{mnum} \mthemnum \label{m13}:] Account Management Module
\end{description}

\begin{table}[h!]
\centering
\begin{tabular}{p{0.3\textwidth} p{0.6\textwidth}}
\toprule
\textbf{Level 1} & \textbf{Level 2}\\
\midrule

\multirow{1}{0.3\textwidth}{Hardware-Hiding Module} 
& Video Input Module \mref{m8}\\
\midrule

\multirow{4}{0.3\textwidth}{Behaviour-Hiding Module} 
& Hand Sign Recognition Module \mref{m1}\\
& Controller Module \mref{m3}\\
& Data Processing Module \mref{m5}\\
& Machine Learning Module \mref{m6}\\
& About/Info Module \mref{m9}\\
& Exercise Module \mref{m10}\\
& Login/Sign Up Module \mref{m12}\\
\midrule

\multirow{3}{0.3\textwidth}{Software Decision Module} 
& Hand Sign Verification Module \mref{m2}\\
& Data Collection Module \mref{m4}\\
& Testing and Verification Module \mref{m7}\\
& Exercise Selection/History Module \mref{m11}\\
& Account Management Module \mref{m13}\\
\bottomrule

\end{tabular}
\caption{Module Hierarchy}
\label{TblMH}
\end{table}

\section{Connection Between Requirements and Design} \label{SecConnection}

The design of the system is intended to satisfy the requirements developed in
the SRS. In this stage, the system is decomposed into modules. The connection
between requirements and modules is listed in Table~\ref{TblRT}.

\section{Module Decomposition} \label{SecMD}

Modules are decomposed according to the principle of ``information hiding''
proposed by \citet{ParnasEtAl1984}. The \emph{Secrets} field in a module
decomposition is a brief statement of the design decision hidden by the
module. The \emph{Services} field specifies \emph{what} the module will do
without documenting \emph{how} to do it. For each module, a suggestion for the
implementing software is given under the \emph{Implemented By} title. If the
entry is \emph{OS}, this means that the module is provided by the operating
system or by standard programming language libraries.  \emph{\progname{}} means the
module will be implemented by the \progname{} software.

Only the leaf modules in the hierarchy have to be implemented. If a dash
(\emph{--}) is shown, this means that the module is not a leaf and will not have
to be implemented.

\subsection{Hardware Hiding Modules}

% \begin{description}
% \item[Secrets:]The data structure and algorithm used to implement the virtual
%   hardware.
% \item[Services:]Serves as a virtual hardware used by the rest of the
%   system. This module provides the interface between the hardware and the
%   software. So, the system can use it to display outputs or to accept inputs.
% \item[Implemented By:] OS
% \end{description}

\subsubsection{Video Input Module (\mref{m8})}
\begin{description}
\item[Secrets:] Image processing algorithms, software interaction with webcam hardware.
\item[Services:] Outputs real-time video data through user's webcam.
\item[Implemented By:] OpenCV and OS.
\end{description}

\subsection{Behaviour-Hiding Modules}

% \begin{description}
% \item[Secrets:]The contents of the required behaviours.
% \item[Services:]Includes programs that provide externally visible behaviour of
%   the system as specified in the software requirements specification (SRS)
%   documents. This module serves as a communication layer between the
%   hardware-hiding module and the software decision module. The programs in this
%   module will need to change if there are changes in the SRS.
% \item[Implemented By:] --
% \end{description}
\subsubsection{Hand Sign Recognition Module (\mref{m1})}
\begin{description}
\item[Secrets:] Recognises hand signs are being made.
\item[Services:] Relays the information that a hand sign is made.
\item[Implemented By:] Python and Pytorch.
\end{description}

\subsubsection{Controller Module (\mref{m3})}
\begin{description}
\item[Secrets:] Able to relay necessary information from back-end component to the correct corresponding front-end component and vice versa.
\item[Services:] Handles communication between front-end components and back-end components.
\item[Implemented By:] Python and JavaScript.
\end{description}

\subsubsection{Data Processing Module (\mref{m5})}
\begin{description}
\item[Secrets:] Algorithm used to process collected data.
\item[Services:] Interprets collected data accordingly.
\item[Implemented By:] Python.
\end{description}

\subsubsection{Machine Learning Module (\mref{m6})}
\begin{description}
\item[Secrets:] Training data sets and structure of neural network.
\item[Services:] Takes in hand coordinate data and interprets and processes the data to return the appropriate letter.
\item[Implemented By:] Python and Pytorch.
\end{description}

\subsubsection{About/Info Module (\mref{m9})}
\begin{description}
\item[Secrets:] Relevant information to shown to user about the program/website.
\item[Services:] Display necessary application information about ASLingo to the user on website.
\item[Implemented By:] Javascript.
\end{description}

\subsubsection{Exercise Module (\mref{m10})}
\begin{description}
\item[Secrets:] Question selection process andn question bank used to quiz user.
\item[Services:] Creates an exercise module with questions based on current users level of progress.
\item[Implemented By:] Javascript.
\end{description}

\subsubsection{Login/Sign Up Module (\mref{m12})}
\begin{description}
\item[Secrets:] How user information is stored and accessed.
\item[Services:] Takes in user input to allow them to sign into their account.
\item[Implemented By:] Javascript.
\end{description}

\subsection{Software Decision Modules}

% \begin{description}
% \item[Secrets:] The design decision based on mathematical theorems, physical
%   facts, or programming considerations. The secrets of this module are
%   \emph{not} described in the SRS.
% \item[Services:] Includes data structure and algorithms used in the system that
%   do not provide direct interaction with the user. 
%   % Changes in these modules are more likely to be motivated by a desire to
%   % improve performance than by externally imposed changes.
% \item[Implemented By:] --
% \end{description}

\subsubsection{Hand Sign Verification Module (\mref{m2})}
\begin{description}
\item[Secrets:] The handshake used to determine whether the sign interpreted by the Machine Learning Module matches with the sign requested by front-end components
\item[Services:] Returns a pass/fail to front-end based on result of match attempt
\item[Implemented By:] Python
\end{description}

\subsubsection{Data Collection Module (\mref{m4})}

\begin{description}
\item[Secrets:] The data and structure of data collected.
\item[Services:] Stores collected data to be retrieved at a later time.
\item[Implemented By:] Python
\end{description}

\subsubsection{Testing and Verification Module (\mref{m7})}
\begin{description}
\item[Secrets:] Unit test cases used to test the software system.
\item[Services:] Verifies the software works as intended by testing the system on various test cases which ensures robustness, accuracy, and reliability.
\item[Implemented By:] Python and Pytest.
\end{description}

\subsubsection{Exercise Selection/History Module (\mref{m11})}
\begin{description}
\item[Secrets:] How historical data is stored for each user.
\item[Services:] Display users completed exercise progress and display current exercise options.
\item[Implemented By:] Javascript.
\end{description}

\subsubsection{Account Management Module (\mref{m13})}
\begin{description}
\item[Secrets:] How user information is stored and processed.
\item[Services:] Will keep a record of users account history after signing in to their account.
\item[Implemented By:] Javascript.
\end{description}

\section{Traceability Matrix} \label{SecTM}

This section shows two traceability matrices: between the modules and the
requirements and between the modules and the anticipated changes.

% the table should use mref, the requirements should be named, use something
% like fref
\begin{table}[H]
\centering
\begin{tabular}{p{0.2\textwidth} p{0.6\textwidth}}
\toprule
\textbf{Req.} & \textbf{Modules}\\
\midrule
FR3 & \mref{m12}\\
FR4 & \mref{m12}\\
FR5 & \mref{m13}\\
FR6 & \mref{m10}\\
FR7 & \mref{m11}\\
FR8 & \mref{m13}\\
FR12 & \mref{m11}\\
FR13 & \mref{m12}\\
LFR1 & \mref{m9}\\
LFR2 & \mref{m11}\\
LFR3 & \mref{m10}\\
UHR3 & \mref{m13}\\
SR1 & \mref{m12}\\
\bottomrule
\end{tabular}
\caption{Trace Between Requirements and Modules}
\label{TblRT}
\end{table}

\begin{table}[H]
\centering
\begin{tabular}{p{0.2\textwidth} p{0.6\textwidth}}
\toprule
\textbf{AC} & \textbf{Modules}\\
\midrule
\acref{acInfo} & \mref{mHH}\\
\acref{acAbout} & \mref{m9}\\
\acref{acExercise} & \mref{m10}, \mref{m11}\\
\acref{acAccount} & \mref{m13}\\
\bottomrule
\end{tabular}
\caption{Trace Between Anticipated Changes and Modules}
\label{TblACT}
\end{table}

\section{Use Hierarchy Between Modules} \label{SecUse}

In this section, the uses hierarchy between modules is
provided. \citet{Parnas1978} said of two programs A and B that A {\em uses} B if
correct execution of B may be necessary for A to complete the task described in
its specification. That is, A {\em uses} B if there exist situations in which
the correct functioning of A depends upon the availability of a correct
implementation of B.  Figure \ref{FigUH} illustrates the use relation between
the modules. It can be seen that the graph is a directed acyclic graph
(DAG). Each level of the hierarchy offers a testable and usable subset of the
system, and modules in the higher level of the hierarchy are essentially simpler
because they use modules from the lower levels.

\begin{figure}[H]
\centering
%\includegraphics[width=0.7\textwidth]{UsesHierarchy.png}
\caption{Use hierarchy among modules}
\label{FigUH}
\end{figure}

%\section*{References}

\bibliographystyle {plainnat}
\bibliography{../../../refs/References}

\newpage{}

\end{document}
