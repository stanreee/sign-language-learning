\documentclass{article}

\usepackage{tabularx}
\usepackage{booktabs}
\usepackage{hyperref}

\title{Problem Statement and Goals\\\progname}

\author{\authname}

\date{}

%% Comments

\usepackage{color}

\newif\ifcomments\commentstrue %displays comments
%\newif\ifcomments\commentsfalse %so that comments do not display

\ifcomments
\newcommand{\authornote}[3]{\textcolor{#1}{[#3 ---#2]}}
\newcommand{\todo}[1]{\textcolor{red}{[TODO: #1]}}
\else
\newcommand{\authornote}[3]{}
\newcommand{\todo}[1]{}
\fi

\newcommand{\wss}[1]{\authornote{blue}{SS}{#1}} 
\newcommand{\plt}[1]{\authornote{magenta}{TPLT}{#1}} %For explanation of the template
\newcommand{\an}[1]{\authornote{cyan}{Author}{#1}}

%% Common Parts

\newcommand{\progname}{Software Engineering} % PUT YOUR PROGRAM NAME HERE
\newcommand{\authname}{Team 15, ASLingo
\\ Andrew Kil
\\ Cassidy Baldin
\\ Edward Zhuang
\\ Jeremy Langner
\\ Stanley Chan} % AUTHOR NAMES                  

\usepackage{hyperref}
    \hypersetup{colorlinks=true, linkcolor=blue, citecolor=blue, filecolor=blue,
                urlcolor=blue, unicode=false}
    \urlstyle{same}
                                


\begin{document}

\maketitle

\begin{table}[hp]
\caption{Revision History} \label{TblRevisionHistory}
\begin{tabularx}{\textwidth}{llX}
\toprule
\textbf{Date} & \textbf{Developer(s)} & \textbf{Change}\\
\midrule
September 19, 2023 & Andrew Kil & Initial Proposed Draft\\
September 20, 2023 & Jeremy Langner & Edited Problem section and ease of access within Goals\\
September 20, 2023 & Edward Zhuang & Edited Problem Statement\\
Date2 & Name(s) & Description of changes\\
... & ... & ...\\
\bottomrule
\end{tabularx}
\end{table}

\section{Problem Statement}


% \wss{You should check your problem statement with the
% \href{https://github.com/smiths/capTemplate/blob/main/docs/Checklists/ProbState-Checklist.pdf}
% {problem statement checklist}.}
% \wss{You can change the section headings, as long as you include the required information.}

\subsection{Problem}

Learning a new language can be an arduous task that only gets more challenging with age, as individuals may find it difficult to dedicate time and effort to it. American Sign Language (ASL) is particularly hard due to its visual and gestural nature, which is not found in other, verbal languages. ASLingo aims to ease that challenge by providing an online, easy-to-access web platform for individuals to learn new signs and test their comprehension at their own pace in a fun, interactive manner. With a focus on consistent effort and continuous feedback, ASLingo provides real-time guidance to ensure users stay on track to achieving their goals of learning ASL.

\subsection{Inputs and Outputs}

The inputs of ASLingo or ``the system" should be video capture of the user's hand gestures which can be interpreted as signs, which the system should be able to determine whether the user is correct or incorrect. The flow from input to output should be:

\begin{itemize}
    \item Into a camera, the user will attempt to form the sign of the system's prompt
    \item The system would process the video capture and determine whether or not the user accurately formed the correct sign
    \item The system would output a response accordingly
\end{itemize}

\subsection{Stakeholders}

The stakeholders for this project includes the hard of hearing and deaf community as well as any individuals who desire to learn ASL. This naturally expands to include educators who wish to promote ASL learning to their respective institutions.

\subsection{Environment}

The software of ASLingo consists of a web app with a computer vision system that is able to translate signs into English. The software will identify hand signs from a live video feed. Hence, the environment will be a computer with a web browser, our computer vision system, and a connected video camera.

\section{Goals}

Ease Of Access:\\
Individuals with a computer than can access Internet should be able to use ASLingo anytime, anywhere. The user's ability to access the system should not be inhibited by anything within the design.

~\\Accessibility:\\
The system should be simple and intuitive to use with little to none required thought from the user to begin the training. Anyone of learning age should be able to utilize ASLingo to learn a few hand signs. 

~\\Engaging\\
A crucial part of the learning experience is keeping the student's attention. ASLingo should be engaging, and thereby fun, enough such that the user is willing to engage with the system independent of external influences.

~\\User Customize Experience\\
A user should be able to focus their learning experience if they wish to learn specific phrases rather than just general signing. The option to specialize what they learn should prove to enrich their experience while using the software.

\section{Stretch Goals}

Mobile Port:\\
Eventually port the web application to a phone application to increase ease-of-access and usability. Having a dedicated mobile experience would naturally improve the Ease of Access and accessibility as accessing web services on mobile is notorious for poor user-end experience.

~\\Expansion to Other Sign Languages\\
ASL is not the only sign language in the world, and since sign language has yet to universally standardized, the integration of different types of sign languages unique to their country of origin would allow for ASLingo to be taken worldwide.

\end{document}