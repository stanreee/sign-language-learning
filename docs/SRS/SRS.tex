\documentclass[12pt, titlepage]{article}

\usepackage{booktabs}
\usepackage{tabularx}
\usepackage{hyperref}
\hypersetup{
    colorlinks,
    citecolor=black,
    filecolor=black,
    linkcolor=red,
    urlcolor=blue
}
\usepackage[round]{natbib}

\title{SE 3XA3: Software Requirements Specification\\Title of Project}

\author{Team \#, Team Name
		\\ Student 1 name and macid
		\\ Student 2 name and macid
		\\ Student 3 name and macid
}

\date{\today}

%% Comments

\usepackage{color}

\newif\ifcomments\commentstrue %displays comments
%\newif\ifcomments\commentsfalse %so that comments do not display

\ifcomments
\newcommand{\authornote}[3]{\textcolor{#1}{[#3 ---#2]}}
\newcommand{\todo}[1]{\textcolor{red}{[TODO: #1]}}
\else
\newcommand{\authornote}[3]{}
\newcommand{\todo}[1]{}
\fi

\newcommand{\wss}[1]{\authornote{blue}{SS}{#1}} 
\newcommand{\plt}[1]{\authornote{magenta}{TPLT}{#1}} %For explanation of the template
\newcommand{\an}[1]{\authornote{cyan}{Author}{#1}}

%% Common Parts

\newcommand{\progname}{Software Engineering} % PUT YOUR PROGRAM NAME HERE
\newcommand{\authname}{Team 15, ASLingo
\\ Andrew Kil
\\ Cassidy Baldin
\\ Edward Zhuang
\\ Jeremy Langner
\\ Stanley Chan} % AUTHOR NAMES                  

\usepackage{hyperref}
    \hypersetup{colorlinks=true, linkcolor=blue, citecolor=blue, filecolor=blue,
                urlcolor=blue, unicode=false}
    \urlstyle{same}
                                


\begin{document}

\maketitle

\pagenumbering{roman}
\tableofcontents
\listoftables
\listoffigures

\begin{table}[bp]
\caption{\bf Revision History}
\begin{tabularx}{\textwidth}{p{3cm}p{2cm}X}
\toprule {\bf Date} & {\bf Version} & {\bf Notes}\\
\midrule
Date 1 & 1.0 & Notes\\
Date 2 & 1.1 & Notes\\
\bottomrule
\end{tabularx}
\end{table}

\newpage

\pagenumbering{arabic}

This document describes the requirements for ASLingo  The template for the Software
Requirements Specification (SRS) is a subset of the Volere
template \textit{Robertson And Robertson (2012)}.  Subsections \textit{Clients} and \textit{Customers} were removed due to not having any such dependents.

\section{Project Drivers}

\subsection{The Purpose of the Project}

Learning a new language can be an arduous task that only gets more challenging
with age, as individuals may find it difficult to dedicate time and effort to
it. American Sign Language (ASL) is particularly hard due to its visual and
gestural nature, which is not found in other, verbal languages. The purpose of this project is
to ease that challenge by providing an online, easy-to-access web platform for
individuals to learn new signs and test their comprehension at their own pace
in a fun, interactive manner. Focusing in on consistent effort and continuous
feedback, ASLingo provides real-time guidance to ensure users stay on track to
achieving their goals of learning ASL.

\subsection{The Stakeholders}

 The stakeholders for this project include those who use sign language as their primary mode of communication in daily life as well as those who have an interest in learning ASL. This would naturally expand outward towards educators who wish to promote the learning of ASL to their respective institutions. 

% \subsubsection{The Client}

% \subsubsection{The Customers}

\subsubsection{Other Stakeholders}

\subsection{Mandated Constraints}

The project is constrained by the following:

\begin{itemize}
    \item The Project Expenses Cannot Exceed \$750
\end{itemize}

\subsection{Naming Conventions and Terminology}

% \hline

% & \\

\noindent \begin{tabularx}{\textwidth}{p{0.3\linewidth}|X}
\toprule
\textbf{Term, Abbreviation, or Acronym} & \textbf{Description}\\
\midrule
A
& Shorthand for Assumption\\
\hline
ASL
& Shorthand for American Sign Language. It is a form of sign language primarily used in the US and in parts of Canada\\
\hline
ASLingo
& The commerical name for the project\\
\bottomrule
\end{tabularx}

\subsection{Relevant Facts and Assumptions}

User characteristics should go under assumptions.

\section{Functional Requirements}

\subsection{The Scope of the Work and the Product}

\subsubsection{The Context of the Work}

\subsubsection{Work Partitioning}

\subsubsection{Individual Product Use Cases}

\subsection{Functional Requirements}

\section{Non-functional Requirements}

\subsection{Look and Feel Requirements}

\subsection{Usability and Humanity Requirements}

\subsection{Performance Requirements}

\subsection{Operational and Environmental Requirements}

\subsection{Maintainability and Support Requirements}

\subsection{Security Requirements}

\subsection{Cultural Requirements}

\subsection{Legal Requirements}

\subsection{Health and Safety Requirements}

This section is not in the original Volere template, but health and safety are
issues that should be considered for every engineering project.

\section{Project Issues}

\subsection{Open Issues}

\subsection{Off-the-Shelf Solutions}

\subsection{New Problems}

\subsection{Tasks}

\subsection{Migration to the New Product}

\subsection{Risks}

\subsection{Costs}

\subsection{User Documentation and Training}

\subsection{Waiting Room}

\subsection{Ideas for Solutions}

\bibliographystyle{plainnat}

\bibliography{SRS}

\newpage

\section{Appendix}

This section has been added to the Volere template.  This is where you can place
additional information.

\subsection{Symbolic Parameters}

The definition of the requirements will likely call for SYMBOLIC\_CONSTANTS.
Their values are defined in this section for easy maintenance.


\end{document}