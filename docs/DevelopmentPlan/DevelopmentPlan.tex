\documentclass{article}

\usepackage{booktabs}
\usepackage{tabularx}

\title{Development Plan\\\progname}

\author{\authname}

\date{}

%% Comments

\usepackage{color}

\newif\ifcomments\commentstrue %displays comments
%\newif\ifcomments\commentsfalse %so that comments do not display

\ifcomments
\newcommand{\authornote}[3]{\textcolor{#1}{[#3 ---#2]}}
\newcommand{\todo}[1]{\textcolor{red}{[TODO: #1]}}
\else
\newcommand{\authornote}[3]{}
\newcommand{\todo}[1]{}
\fi

\newcommand{\wss}[1]{\authornote{blue}{SS}{#1}} 
\newcommand{\plt}[1]{\authornote{magenta}{TPLT}{#1}} %For explanation of the template
\newcommand{\an}[1]{\authornote{cyan}{Author}{#1}}

%% Common Parts

\newcommand{\progname}{Software Engineering} % PUT YOUR PROGRAM NAME HERE
\newcommand{\authname}{Team 15, ASLingo
\\ Andrew Kil
\\ Cassidy Baldin
\\ Edward Zhuang
\\ Jeremy Langner
\\ Stanley Chan} % AUTHOR NAMES                  

\usepackage{hyperref}
    \hypersetup{colorlinks=true, linkcolor=blue, citecolor=blue, filecolor=blue,
                urlcolor=blue, unicode=false}
    \urlstyle{same}
                                


\begin{document}

\maketitle

\begin{table}[hp]
\caption{Revision History} \label{TblRevisionHistory}
\begin{tabularx}{\textwidth}{llX}
\toprule
\textbf{Date} & \textbf{Developer(s)} & \textbf{Change}\\
\midrule
September 22, 2023 & Stanley Chan & Proposed proof of concept demonstration plan\\
September 22, 2023 & Andrew Kil & Proposed workflow plan\\
September 22, 2023 & Edward Zhuang & Proposed some initial stage technology suggestions\\
September 22, 2023 & Everyone & Finished the Rest of the Dev Plan\\
\bottomrule
\end{tabularx}
\end{table}

% \wss{Put your introductory blurb here.}

\section{Team Meeting Plan}

As a baseline, when there is no lecture or tutorial for this course, all group members should aim to meet together. All meetings should occur at a frequency of a least once a week. Otherwise, members will use the When2Meet schedule planner to determine an ideal time slot for meetings outside of the previously mentioned times. Team meetings should always try to accommodate all members and should occur over voice call using the Discord platform.

\section{Team Communication Plan}

All team communications will primarily be done using Discord and when meeting in-person in lectures. Communication regarding project issues is encouraged to be done through raising GitHub issues so that comments are directly attached to the codes they concern. When communication with any supervisors, stakeholders, TA's or Instructors, it must be done in email with all member's CC'ed to the chain.

\section{Team Member Roles}

All members will be responsible for documentation, general testing, code review, and stakeholder outreach. 

\begin{table}%[H]
\caption{Team Roles} \label{TblRoles}
\begin{tabularx}{\textwidth}{ll}
\toprule
\textbf{Name} & \textbf{Responsibilities}\\
\midrule
Andrew Kil & Testing, UI/UX, Computer vision component\\
Cassidy Baldin & Full stack, sign language learning\\
Edward Zhuang & Backend development, Computer vision component\\
Jeremy Langner & Full stack development\\
Stanley Chan & Computer vision component, Frontend component\\
\bottomrule
\end{tabularx}
\end{table}

\section{Workflow Plan}

% \begin{itemize}
% 	\item How will you be using git, including branches, pull request, etc.?
% 	\item How will you be managing issues, including template issues, issue
% 	classificaiton, etc.?
% \end{itemize}
GitHub will be used to manage the project. The general structure of development will be as follows. Development branches called "Milestone Branches" will be opened at the start of a new deliverable. All members must then ensure that they are up to date on all the latest documents before syncing to the Milestone Branch. Once synced, members will work on that branch until that Milestone is reached. Final tests will occur to ensure no unintended errors, and after a group agreement, the Milestone Branch will be merged into main. When being merged, it must be done so with a comment on the changes made, and pushed so that all members can see the new documentation.

GitHub's integrated Issue Tracking feature will be used for all issue management as well as for keeping track of the current state of work.  Issues will be raised via this feature whenever a new issue type like "bugs" occurs. All future issues with the same classification will work off this template.

\section{Proof of Concept Demonstration Plan}

One significant risk we have identified pertains to the accuracy and reliability of the computer vision aspect of this application. As this is the
primary functionality for this application, it is necessary that our system should demonstrate the capability to detect and recognize complex
hand signs and motions. One way we could validate our ability to mitigate this risk would be to prove that we can build a 
system that can accurately recognize all the letters of the American Sign Language alphabet.
\\

Another risk relates to ensuring that the system can provide realtime feedback based on the user's hand sign inputs with
an acceptable time delay. As this will be a web application, we suspect that much of the computer vision computation will be on the 
server/backend side. This may introduce uncertainty in creating a realtime feedback system, particularly since computer vision is more computationally
intensive compared to text recognition based applications like Duolingo. For our proof of concept, we will deploy our computer vision model
onto a server, and test the delay in the server's response when it receives a request to recognize a hand sign from another device.

\section{Technology}

\begin{itemize}
\item JavaScript, frontend development
\item React, frontend development
\item Python, backend development
\item Flask, backend development
\item Flake8, Python linter
\item OpenCV, computer vision library
\item A camera, live video capture
\item Untit Testing: pytest (Python), Jest (JavaScript)
\item Code Coverage Tools: Coverage.py (Python), Istanbul (JavaScript)
\item GitHub Actions for Continuous Integration and branch protection
\item Libraries: OpenCV, pandas, numpy, Tailwind CSS, Bootstrap, React Router, Redux
\item Tools: Git, Overleaf, VSCode
% \item Specific programming language
% \item Specific linter tool (if appropriate)
% \item Specific unit testing framework
% \item Investigation of code coverage measuring tools
% \item Specific plans for Continuous Integration (CI), or an explanation that CI
%   is not being done
% \item Specific performance measuring tools (like Valgrind), if
%   appropriate
% \item Libraries you will likely be using?
% \item Tools you will likely be using?
\end{itemize}

\section{Coding Standard}

Coding standards will be language specific with all code in Python being upheld to the PEP8 coding standard and all code written in JavaScript upheld to the AirBnB style guide.

\section{Project Scheduling}

Refer to the schedule within the course outline. In general scheduling will ideally be done one deliverable in advance.

% \wss{How will the project be scheduled?}

\end{document}