\documentclass[12pt, titlepage]{article}

\usepackage{amsmath, mathtools}

\usepackage[round]{natbib}
\usepackage{amsfonts}
\usepackage{amssymb}
\usepackage{graphicx}
\usepackage{colortbl}
\usepackage{xr}
\usepackage{hyperref}
\usepackage{longtable}
\usepackage{xfrac}
\usepackage{tabularx}
\usepackage{float}
\usepackage{siunitx}
\usepackage{booktabs}
\usepackage{multirow}
\usepackage[section]{placeins}
\usepackage{caption}
\usepackage{fullpage}

\hypersetup{
bookmarks=true,     % show bookmarks bar?
colorlinks=true,       % false: boxed links; true: colored links
linkcolor=red,          % color of internal links (change box color with linkbordercolor)
citecolor=blue,      % color of links to bibliography
filecolor=magenta,  % color of file links
urlcolor=cyan          % color of external links
}

\usepackage{array}

\externaldocument{../../SRS/SRS}

%% Comments

\usepackage{color}

\newif\ifcomments\commentstrue %displays comments
%\newif\ifcomments\commentsfalse %so that comments do not display

\ifcomments
\newcommand{\authornote}[3]{\textcolor{#1}{[#3 ---#2]}}
\newcommand{\todo}[1]{\textcolor{red}{[TODO: #1]}}
\else
\newcommand{\authornote}[3]{}
\newcommand{\todo}[1]{}
\fi

\newcommand{\wss}[1]{\authornote{blue}{SS}{#1}} 
\newcommand{\plt}[1]{\authornote{magenta}{TPLT}{#1}} %For explanation of the template
\newcommand{\an}[1]{\authornote{cyan}{Author}{#1}}

%% Common Parts

\newcommand{\progname}{Software Engineering} % PUT YOUR PROGRAM NAME HERE
\newcommand{\authname}{Team 15, ASLingo
\\ Andrew Kil
\\ Cassidy Baldin
\\ Edward Zhuang
\\ Jeremy Langner
\\ Stanley Chan} % AUTHOR NAMES                  

\usepackage{hyperref}
    \hypersetup{colorlinks=true, linkcolor=blue, citecolor=blue, filecolor=blue,
                urlcolor=blue, unicode=false}
    \urlstyle{same}
                                


\begin{document}

\title{Module Interface Specification for \progname{}}

\author{\authname}

\date{\today}

\maketitle

\pagenumbering{roman}

\section{Revision History}

\begin{tabularx}{\textwidth}{p{3cm}p{2cm}X}
\toprule {\bf Date} & {\bf Version} & {\bf Notes}\\
\midrule
Jan 16, 2024 & 1.0 & Andrew, Stan, Edward; Finished back-end MIS breakdown\\
Jan 17, 2024 & 1.1 & Jeremy, Cassidy; Finished front-end MIS breakdown, fixed some formatting \\
Apr 3, 2024 & 1.2 & Cassidy; Removed unnecessary modules and cleaned up others \\
Apr 4, 2024 & 1.3 & Jeremy and Cassidy; Added in Quiz Creation Module and other updates \\
\bottomrule
\end{tabularx}

~\newpage

\section{Symbols, Abbreviations and Acronyms}

See SRS Documentation at \href{https://github.com/stanreee/sign-language-learning/blob/main/docs/SRS/SRS.pdf}{Software Requirements Specification} \\

\renewcommand{\arraystretch}{1.2}
\begin{tabular}{|p{3cm}|p{13cm}|} 
  \toprule		
  \textbf{symbol} & \textbf{description}\\
  \midrule 
  AC & Anticipated Change\\
  ASL & Shorthand for American Sign Language. It is a form of sign language primarily used in the US and in parts of Canada. \\
  ASLingo & The commercial name for the project. \\
  CV & Refers to Computer Vision, the field of technology that invloves processing visual input ot achieve various means. \\
  HSR & Shorthand for "Health and Safety Requirements", a subsection of Non-Functional Requirements. \\
  FR & Shorthand for Functional Requirements. \\
  LR & Shorthand for "Legal Requirements", a subsection of Non-Functional Requirements. \\
  LFR & Shorthand for "Look and Feel Requirements", a subsection of Non-Functional Requirements. \\
  MSR & Shorthand for "Maintainability and Support Requirements", a subsection of Non-Functional Requirements. \\
  OER & Shorthand for "Operational and Environmental Requirements", a subsection of Non-Functional Requirements. \\
  OpenCV & Refers to the Open Computer Vision Library library available for free to developers in order to develop Computer Vision applications. \\
  M & Module \\
  MG & Module Guide \\
  PR & Shorthand for "Performance Requirements", a subsection of Non-Functional Requirements. \\
  SR & Shorthand for "Security Requirements", a subsection of Non-Functional Requirements. \\
  SRS & Software Requirements Specification\\
  UC & Unlikely Change \\
  UHR & Shorthand for "Usuability and Humanity Requirements", a subsection of Non-Functional Requirements. \\
  \bottomrule
\end{tabular}\\

\newpage

\tableofcontents

\newpage

\pagenumbering{arabic}

\section{Introduction}

The following document details the Module Interface Specifications for our project ASLingo. 
Learning a new language can be an arduous task that only gets more challenging
with age, as individuals may find it difficult to dedicate time and effort to
it. American Sign Language (ASL) is particularly hard due to its visual and
gestural nature, which is not found in other, verbal languages. The purpose of this project is
to ease that challenge by providing an online, easy-to-access web platform for
individuals to learn new signs and test their comprehension at their own pace
in a fun, interactive manner. Focusing in on consistent effort and continuous
feedback, ASLingo provides real-time guidance to ensure users stay on track to
achieving their goals of learning ASL. \\

Complementary documents include the  \href{https://github.com/stanreee/sign-language-learning/blob/main/docs/SRS/SRS.pdf}{Software Requirements Specification} and \href{https://github.com/stanreee/sign-language-learning/blob/DesignDocRev0/docs/Design/SoftArchitecture/MG.pdf}{Module Guide}.  
The full documentation and implementation can be
found here: \href{https://github.com/stanreee/sign-language-learning/tree/main}{ASLingo Github Repo}.

\section{Notation}

The structure of the MIS for modules comes from Hoffman And Strooper 1995,
with the addition that template modules have been adapted from
Ghezzi Et Al 2003.  The mathematical notation comes from Chapter 3 of
Hoffman And Strooper 1995.  For instance, the symbol := is used for a
multiple assignment statement and conditional rules follow the form $(c_1
\Rightarrow r_1 | c_2 \Rightarrow r_2 | ... | c_n \Rightarrow r_n )$.

The following table summarizes the primitive data types used by \progname. 

\begin{center}
\renewcommand{\arraystretch}{1.2}
\noindent 
\begin{tabular}{l l p{7.5cm}} 
\toprule 
\textbf{Data Type} & \textbf{Notation} & \textbf{Description}\\ 
\midrule
character & char & a single symbol or digit\\
integer & $\mathbb{Z}$ & a number without a fractional component in (-$\infty$, $\infty$) \\
natural number & $\mathbb{N}$ & a number without a fractional component in [1, $\infty$) \\
real & $\mathbb{R}$ & any number in (-$\infty$, $\infty$)\\
\bottomrule
\end{tabular} 
\end{center}

\noindent
The specification of \progname \ uses some derived data types: sequences, strings, and
tuples. Sequences are lists filled with elements of the same data type. Strings
are sequences of characters. Tuples contain a list of values, potentially of
different types. In addition, \progname \ uses functions, which
are defined by the data types of their inputs and outputs. Local functions are
described by giving their type signature followed by their specification.

\section{Module Decomposition}

The following table is taken directly from the \href{https://github.com/stanreee/sign-language-learning/blob/DesignDocRev0/docs/Design/SoftArchitecture/MG.pdf}{Module Guide} document for this project.

\begin{table}[h!]
\centering
\begin{tabular}{p{0.3\textwidth} p{0.6\textwidth}}
\toprule
\textbf{Level 1} & \textbf{Level 2}\\
\midrule

\multirow{1}{0.3\textwidth}{Hardware-Hiding Module} 
& Video Input Module\\
\midrule

\multirow{4}{0.3\textwidth}{Behaviour-Hiding Module} 
& Hand Sign Recognition Module\\
& Controller Module \\
& Data Processing Module \\
& Machine Learning Module \\
& Landing Page Module \\
& Exercise Module \\
& Exercise Selection Module \\
\midrule

\multirow{3}{0.3\textwidth}{Software Decision Module} 
& Hand Sign Verification Module \\
& Data Collection Module \\
& Testing and Verification Module \\
& Quiz Creation Module \\
\bottomrule

\end{tabular}
\caption{Module Hierarchy}
\label{TblMH}
\end{table}

\newpage

% ---------------Template------------------
% \section{MIS of \wss{Module Name}} \label{Module} \wss{Use labels for
%   cross-referencing}

% \wss{You can reference SRS labels, such as R\ref{R_Inputs}.}

% \wss{It is also possible to use \LaTeX for hypperlinks to external documents.}

% \subsection{Module}

% \wss{Short name for the module}

% \subsection{Uses}


% \subsection{Syntax}

% \subsubsection{Exported Constants}

% \subsubsection{Exported Access Programs}

% \begin{center}
% \begin{tabular}{p{2cm} p{4cm} p{4cm} p{2cm}}
% \hline
% \textbf{Name} & \textbf{In} & \textbf{Out} & \textbf{Exceptions} \\
% \hline
% \wss{accessProg} & - & - & - \\
% \hline
% \end{tabular}
% \end{center}

% \subsection{Semantics}

% \subsubsection{State Variables}

% \wss{Not all modules will have state variables.  State variables give the module
%   a memory.}
% \subsubsection{Environment Variables}
% \wss{This section is not necessary for all modules.  Its purpose is to capture
%   when the module has external interaction with the environment, such as for a
%   device driver, screen interface, keyboard, file, etc.}
% \subsubsection{Assumptions}
% \wss{Try to minimize assumptions and anticipate programmer errors via
%   exceptions, but for practical purposes assumptions are sometimes appropriate.}
% \subsubsection{Access Routine Semantics}
% \noindent \wss{accessProg}():
% \begin{itemize}
% \item transition: \wss{if appropriate} 
% \item output: \wss{if appropriate} 
% \item exception: \wss{if appropriate} 
% \end{itemize}
% \wss{A module without environment variables or state variables is unlikely to
%   have a state transition.  In this case a state transition can only occur if
%   the module is changing the state of another module.}
% \wss{Modules rarely have both a transition and an output.  In most cases you
%   will have one or the other.}
% \subsubsection{Local Functions}
% \wss{As appropriate} \wss{These functions are for the purpose of specification.
%   They are not necessarily something that is going to be implemented
%   explicitly.  Even if they are implemented, they are not exported; they only
%   have local scope.}
% ---------------Template------------------

\newpage

% --------------Hand Sign Recog------------------
\section{MIS of Hand Sign Recognition Module} \label{Module} 
\subsection{Module}
HSR
\subsection{Uses}
Machine Learning Module, Video Input Module
\subsection{Syntax}
% \subsubsection{Exported Constants}
\subsubsection{Exported Access Programs}
\begin{center}
\begin{tabular}{p{4cm} p{2cm} p{2cm} p{4cm}}
\hline
\textbf{Name} & \textbf{In} & \textbf{Out} & \textbf{Exceptions} \\
\hline
determine\_handsign & - & String & TIME\_LIMIT\_REACHED \\
\hline
\end{tabular}
\end{center}
\subsection{Semantics}
\subsubsection{State Variables}
\begin{itemize}
    \item MAX\_DECISION\_FRAMES - Frames needed to determine when the user has settled on a hand sign
    \item TIMEOUT\_LIMIT - Amount of time in seconds before the user automatically fails the question
\end{itemize}
\subsubsection{Environment Variables}
None
\subsubsection{Assumptions}
None
\subsubsection{Access Routine Semantics}
\noindent determine\_handsign():
\begin{itemize}
% \item transition: \wss{if appropriate} 
\item output: The name of the determined handsign
\item exception: exc := TIME\_LIMIT\_REACHED
\end{itemize}
\subsubsection{Local Functions}
process\_features()
% --------------Hand Sign Recog------------------

\newpage

% --------------Hand Sign Verif------------------
\section{MIS of Hand Sign Verification Module} \label{Module} 
\subsection{Module}
HSV
\subsection{Uses}
Hand Sign Recognition Module, Controller
\subsection{Syntax}
 \subsubsection{Exported Constants}
None
\subsubsection{Exported Access Programs}
\begin{center}
\begin{tabular}{p{4cm} p{4cm} p{4cm} p{2cm}}
\hline
\textbf{Name} & \textbf{In} & \textbf{Out} & \textbf{Exceptions} \\
\hline
verify\_handsign & - & Boolean & - \\
\hline
\end{tabular}
\end{center}
\subsection{Semantics}
\subsubsection{State Variables}
None
\subsubsection{Environment Variables}
None
\subsubsection{Assumptions}
None
\subsubsection{Access Routine Semantics}
\noindent verify\_handsign():
\begin{itemize}
\item output: True/False for if the expected handsign was made 
\item exception: exc := None 
\end{itemize}
 \subsubsection{Local Functions}
None

% --------------Hand Sign Verif------------------

\newpage

% ----------------Controller--------------------
\section{MIS of Controller Module} \label{Module} 
\subsection{Module}
Controller
\subsection{Uses}
Exercise Module, Hand Sign Verification Module
\subsection{Syntax}
 \subsubsection{Exported Constants}
None
\subsubsection{Exported Access Programs}
\begin{center}
\begin{tabular}{p{5cm} p{2cm} p{2cm} p{2cm}}
\hline
\textbf{Name} & \textbf{In} & \textbf{Out} & \textbf{Exceptions} \\
\hline
send\_requested\_handsign & String & None & - \\
get\_requested\_handsign & None & String & - \\
send\_passFail & Bool & None & - \\
get\_passFail & None & Bool & - \\
\hline
\end{tabular}
\end{center}
\subsection{Semantics}
\subsubsection{State Variables}
\begin{itemize}
    \item h - handsign variable to store intermediary data
    \item pass - Boolean to determine if the question was answered correctly
\end{itemize}
\subsubsection{Environment Variables}
None
\subsubsection{Assumptions}
None
\subsubsection{Access Routine Semantics}
\noindent send\_requested\_handsign():
\begin{itemize}
\item output: None 
\item exception: exc := None
\end{itemize}

\noindent get\_requested\_handsign():
\begin{itemize}
\item output: The expected handsign being asked by the front-end
\item exception: exc := None
\end{itemize}

\noindent send\_passFail():
\begin{itemize}
\item output: None
\item exception: exc := None
\end{itemize}

\noindent get\_passFail():
\begin{itemize}
\item output: The result of comparing the expected answer to what the back-end determined
\item exception: exc := None
\end{itemize}
 \subsubsection{Local Functions}
None
% ----------------Controller--------------------

\newpage

% --------------Data Collection------------------
\section{MIS of Data Collection Module} \label{Module} 
\subsection{Module}
DCM
\subsection{Uses}
None
\subsection{Syntax}
\subsubsection{Exported Constants}
None
\subsubsection{Exported Access Programs}
\begin{center}
\begin{tabular}{p{4cm} p{5cm} p{2cm} p{2cm}}
\hline
\textbf{Name} & \textbf{In} & \textbf{Out} & \textbf{Exceptions} \\
\hline
read\_training\_set & training\_imgs\_path & - & - \\
\hline
\end{tabular}
\end{center}
\subsection{Semantics}
\subsubsection{State Variables}
None
\subsubsection{Environment Variables}
None
\subsubsection{Assumptions}
None
\subsubsection{Access Routine Semantics}
\noindent read\_training\_set():
\begin{itemize}
\item transition: training.csv updated with raw training data
\item output: none
\item exception: exc := None 
\end{itemize}
\subsubsection{Local Functions}
None
% --------------Data Collection------------------

\newpage

% --------------Data Processing------------------
\section{MIS of Data Processing Module} \label{Module} 
\subsection{Module}
DPM
\subsection{Uses}
Data Collection Module
\subsection{Syntax}
\subsubsection{Exported Constants}
\subsubsection{Exported Access Programs}
\begin{center}
\begin{tabular}{p{5cm} p{3cm} p{2cm} p{2cm}}
\hline
\textbf{Name} & \textbf{In} & \textbf{Out} & \textbf{Exceptions} \\
\hline
process\_training\_data & training.csv & - & - \\
\hline
\end{tabular}
\end{center}
\subsection{Semantics}
\subsubsection{State Variables}
None
\subsubsection{Environment Variables}
None
\subsubsection{Assumptions}
None
\subsubsection{Access Routine Semantics}
\noindent process\_training\_data():
\begin{itemize}
    \item transition: training.csv updated with processed training data
    \item output: none
    \item exception: exc := None
\end{itemize}
\subsubsection{Local Functions}
None
% --------------Data Processing------------------

\newpage

% --------------Machine Learning------------------
\section{MIS of Machine Learning Module} \label{Module} 
\subsection{Module}
MLM
\subsection{Uses}
Data Processing Module
\subsection{Syntax}
\subsubsection{Exported Constants}
None
\subsubsection{Exported Access Programs}
\begin{center}
\begin{tabular}{p{4cm} p{5cm} p{2cm} p{2cm}}
\hline
\textbf{Name} & \textbf{In} & \textbf{Out} & \textbf{Exceptions} \\
\hline
SignLangModel & processed\_frame\_data & String & - \\
\hline
\end{tabular}
\end{center}
\subsection{Semantics}
\subsubsection{State Variables}
None
\subsubsection{Environment Variables}
None
\subsubsection{Assumptions}
None
\subsubsection{Access Routine Semantics}
SignLangModel():
\begin{itemize}
    \item transition: none
    \item output: Predicted hand sign for given processed frame data
    \item exception: none
\end{itemize}
\subsubsection{Local Functions}
train()
% --------------Machine Learning------------------

\newpage

% --------------Test-Verif------------------
\section{MIS of Testing and Verification Module} \label{Module} 
\subsection{Module}
Tester
\subsection{Uses}
Hand Sign Verification Module
\subsection{Syntax}
 \subsubsection{Exported Constants}
 None
 \subsubsection{Exported Access Programs}
 None
\subsection{Semantics}
\subsubsection{State Variables}
None
\subsubsection{Environment Variables}
None
\subsubsection{Assumptions}
None
\subsubsection{Access Routine Semantics}
None
\subsubsection{Local Functions}
testDataCollectionModule()\\
testDataProcessingModule()\\
testMachineLearningModule()\\
testVideoInputModule()\\
testHandSignRecognitionModule()\\
testHandSignVerificationModule()\\
% --------------Test-Verif------------------

\newpage

% --------------Video Input------------------
\section{MIS of Video Input Module} \label{Module} 
\subsection{Module}
Cam
\subsection{Uses}
None
\subsection{Syntax}
\subsubsection{Exported Access Programs}
\begin{center}
\begin{tabular}{p{4cm} p{3cm} p{4cm} p{2cm}}
\hline
\textbf{Name} & \textbf{In} & \textbf{Out} & \textbf{Exceptions} \\
\hline
get\_frame\_data & video input & NumPy ndarray & - \\
\hline
\end{tabular}
\end{center}
\subsection{Semantics}
\subsubsection{State Variables}
None
\subsubsection{Environment Variables}
None
\subsubsection{Assumptions}
None
\subsubsection{Access Routine Semantics}
get\_frame\_data():
\begin{itemize}
    \item transition: raw video input is turned into an array of shape (height, width, channels)
    \item output: the frame data read through the video feed in terms of a NumPy array
    \item exception: exc := None
\end{itemize}

% --------------Video Input------------------

\newpage

% --------------Landing Page------------------

\section{MIS of Landing Page Module} \label{Module} 

\subsection{Module}

Landing Page Module.

\subsection{Uses}

None

\subsection{Syntax}

\subsubsection{Exported Constants}

None

\subsubsection{Exported Access Programs}

\begin{center}
\begin{tabular}{p{4cm} p{4cm} p{2cm} p{2cm}}
\hline
\textbf{Name} & \textbf{In} & \textbf{Out} & \textbf{Exceptions} \\
\hline
getAboutInfo & - & String & - \\
getInstructions & - & String & - \\
\hline
\end{tabular}
\end{center}

\subsection{Semantics}

\subsubsection{State Variables}

\begin{itemize}
    \item aboutInfo: string
    \item instructionInfo: string
\end{itemize}

\subsubsection{Assumptions}

None

\subsubsection{Access Routine Semantics}

\noindent getAboutInfo():
\begin{itemize}
\item transition: 
\item output: aboutInfo
\item exception: 
\end{itemize}

\noindent getInstructions():
\begin{itemize}
\item transition:
\item output: instructionInfo
\item exception: 
\end{itemize}


\subsubsection{Local Functions}

None

% --------------Landing Page------------------

\newpage

% --------------Exercise------------------

\section{MIS of Exercise Module} \label{Exercise} 
% module that feeds into the exercise selection module which contains all the exercises a user might encounter
% when using the program. This includes things like difficulty level, type of exercise, question, answer
% depending on which type of exercise it is the ans may be text or CV based), score

\subsection{Module}

Exercise

\subsection{Uses}

Controller Module

\subsection{Syntax}

\subsubsection{Exported Constants}

None

\subsubsection{Exported Access Programs}

% difficulty level, type of exercise, question, answer
% depending on which type of exercise it is the ans may be text or CV based), score

\begin{center}
\begin{tabular}{p{4cm} p{3cm} p{3cm} p{4cm}}
\hline
\textbf{Name} & \textbf{In} & \textbf{Out} & \textbf{Exceptions} \\
\hline
getQuestions & - & string & \\
getAnswers & - & string & - \\
getQuestionDifficulty & - & int &  \\
getQuestionScores & - & int & \\
getExerciseTotalScore & - & int & \\
\hline
\end{tabular}
\end{center}

\subsection{Semantics}

\subsubsection{State Variables}

\begin{itemize}
    \item title :  string
    \item description :  string
    \item questionAmount: int
    \item timePerQuestion: int
    \item questions : seq of string % list of questions to ask the user \\
    \item answers : seq of string % list of answers that correspond to the same question instance \\
    \item difficultyLevels :  string of `easy', `medium', `hard'
    \item questionScores : seq of $\mathbb{Z}$  % list of scores the user gets on questions in the exercise, starts empty, scores are passed to exercise history module    \item exerciseTotalScore : $\mathbb{Z}$, initialized to 0 at the start of each exercise % variable to keep track of the users current total score in the exercise
\end{itemize}

\subsubsection{Environment Variables}

\begin{itemize}
    \item userAnswer := string % result of the user's signed answer from hand sign recognition module
\end{itemize}

\subsubsection{Assumptions}

None 

\subsubsection{Access Routine Semantics}

\noindent getQuestions():
\begin{itemize}
\item output: questions := seq of string
\item exception: None
\end{itemize}

\noindent getAnswers():
\begin{itemize}
\item output: answers := seq of string
\item exception: None
\end{itemize}

\noindent getQuestionDifficulty():
\begin{itemize}
\item output: difficultyLevels := seq of $\mathbb{N}$
\item exception: None
\end{itemize}

\noindent getQuestionScores():
\begin{itemize}
\item output: questionScores := seq of $\mathbb{Z}$
\item exception: None
\end{itemize}

\noindent getExerciseTotalScore():
\begin{itemize}
\item transition: exerciseTotalScore variable is updated after the user answers the question given to them in the exercise to be displayed at the end of each quiz.
\item output: exerciseTotalScore := seq of $\mathbb{Z}$
\item exception: None
\end{itemize}
%
\subsubsection{Local Functions}

updateTotalScore: exerciseTotalScore $\rightarrow$ exerciseTotalScore + givenScore \\
givenScore: type int ($\mathbb{Z}$), [0, numOfQuestions]

% --------------Exercise------------------

\newpage

% --------------Exercise Selection History ------------------

\section{MIS of Exercise Selection Module} \label{ExerciseSel} 

\subsection{Module}

Exercise Selection

\subsection{Uses}

Exercise Module

\subsection{Syntax}

\subsubsection{Exported Constants}

None

\subsubsection{Exported Access Programs}

% contains a way to choose a list of questions and answers for the user based on their account history and other exercises they have completed
% the history will keep track of how well they did on a question and track their scores

\begin{center}
\begin{tabular}{p{6cm} p{2cm} p{2cm} p{4cm}}
\hline
\textbf{Name} & \textbf{In} & \textbf{Out} & \textbf{Exceptions} \\
\hline
createQuestionList & - & string & - \\
createAnswerList & string & string & - \\
updateExerciseQuestionHistory & string & string & InvalidEntry \\  % keeps a list of questions they have been asked
updateExerciseAnswerHistory & string & string & InvalidEntry \\   % keeps a list of answers the user has given to a corresponding question
\hline
\end{tabular}
\end{center}

\subsection{Semantics}

\subsubsection{State Variables}

\begin{itemize}
    \item questionList := seq of string
    \item answerList := seq of string
    \item exerciseAnswer := string 
    \item exerciseScore := $\mathbb{Z}$ 
\end{itemize}

\subsubsection{Environment Variables}

None 

\subsubsection{Assumptions}

None 

\subsubsection{Access Routine Semantics}

\noindent createQuestionList(): 
\begin{itemize}
\item output: questionList := seq of string
\item exception: None
\end{itemize}

\noindent createAnswerList(): 
\begin{itemize}
\item output: answerList := seq of string
\item exception: None
\end{itemize}


\noindent updateExerciseAnswer(): 
\begin{itemize}
\item output:exerciseAnswer := string \\
\item exception: None
\end{itemize}

\noindent updateExerciseScore(): 
\begin{itemize}
\item output: exerciseScore := $\mathbb{Z}$ \\
\item exception: None
\end{itemize}

\subsubsection{Local Functions}

None

% --------------Exercise Selection History ------------------

\newpage

% --------------Quiz Creation------------------
\section{MIS of Quiz Creation Module} \label{Module} 
\subsection{Module}
Quiz Creation
\subsection{Uses}
Exercise Selection Module
\subsection{Syntax}
 \subsubsection{Exported Constants}
 None
 \subsubsection{Exported Access Programs}

    \begin{center}
    \begin{tabular}{p{4cm} p{3cm} p{3cm} p{4cm}}
    \hline
    \textbf{Name} & \textbf{In} & \textbf{Out} & \textbf{Exceptions} \\
    \hline
    getEasyQuizzes & - & iQuiz[] & \\
    getMediumQuizzes & - & iQuiz[] & \\
    getHardQuizzes & - & iQuiz[] &  \\
    getVowelQuizzes & - & iQuiz[] & \\
    getEasyStaticQuizzes & - & iQuiz[] & \\
    getMediumStaticQuizzes & - & iQuiz[] & \\
    getMediumDynamicQuizzes & - & iQuiz[] &  \\
    getHardDynamicQuizzes & - & iQuiz[] & \\
    \hline
    \end{tabular}
    \end{center}


 
\subsection{Semantics}
\subsubsection{State Variables}
None
\subsubsection{Environment Variables}
None
\subsubsection{Assumptions}
None
\subsubsection{Access Routine Semantics}

\noindent getEasyQuizzes():
\begin{itemize}
\item output: questions := seq of iQuiz
\item exception: None
\end{itemize}

\noindent getMediumQuizzes():
\begin{itemize}
\item output: questions := seq of iQuiz
\item exception: None
\end{itemize}

\noindent getHardQuizzes():
\begin{itemize}
\item output: questions := seq of iQuiz
\item exception: None
\end{itemize}

\noindent getEasyStaticQuizzes():
\begin{itemize}
\item output: questions := seq of iQuiz
\item exception: None
\end{itemize}

\noindent getMediumStaticQuizzes():
\begin{itemize}
\item output: questions := seq of iQuiz
\item exception: None
\end{itemize}

\noindent getMediumDynamicQuizzes():
\begin{itemize}
\item output: questions := seq of iQuiz
\item exception: None
\end{itemize}

\noindent getHardDynamicQuizzes():
\begin{itemize}
\item output: questions := seq of iQuiz
\item exception: None
\end{itemize}

\subsubsection{Local Functions}
getQuizzes(numQuizzes: int, difficulty: string, questions: int, timePerQuestion: int, description: string, title:string)\\
selectRandomQuestions(difficulty: string, numQuestions: int)\\
% --------------Test-Verif------------------

\newpage

%\bibliographystyle {plainnat}
%\bibliography {../../../refs/References}

%\newpage
%
%\section{Appendix} \label{Appendix}
%
%\wss{Extra information if required}

\end{document}