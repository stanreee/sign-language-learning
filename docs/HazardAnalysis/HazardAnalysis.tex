\documentclass{article}

\usepackage{booktabs}
\usepackage{tabularx}
\usepackage{hyperref}
\usepackage{tabularray}
\usepackage{longtable}
\usepackage[hmargin=1cm]{geometry}
\usepackage{float}


\hypersetup{
    colorlinks=true,       % false: boxed links; true: colored links
    linkcolor=red,          % color of internal links (change box color with linkbordercolor)
    citecolor=green,        % color of links to bibliography
    filecolor=magenta,      % color of file links
    urlcolor=cyan           % color of external links
}

\title{Hazard Analysis\\\progname}

\author{\authname}

\date{}

\input{../Comments}
%% Common Parts

\newcommand{\progname}{Software Engineering} % PUT YOUR PROGRAM NAME HERE
\newcommand{\authname}{Team 15, ASLingo
\\ Andrew Kil
\\ Cassidy Baldin
\\ Edward Zhuang
\\ Jeremy Langner
\\ Stanley Chan} % AUTHOR NAMES                  

\usepackage{hyperref}
    \hypersetup{colorlinks=true, linkcolor=blue, citecolor=blue, filecolor=blue,
                urlcolor=blue, unicode=false}
    \urlstyle{same}
                                


\begin{document}

\maketitle
\thispagestyle{empty}

~\newpage

\pagenumbering{roman}

\begin{table}[hp]
\caption{Revision History} \label{TblRevisionHistory}
\begin{tabularx}{\textwidth}{llX}
\toprule
\textbf{Date} & \textbf{Developer(s)} & \textbf{Change}\\
\midrule
Oct 17 2022 & Jeremy & Added 2 FMEA table entries related to web application hazards\\
Oct 17 2022 & Andrew & Added Camera FMEA table entry\\
Oct 17 2022 & Jeremy & Added 2 more FMEA table entries related to web application\\
Date1 & Name(s) & Description of changes\\
Date2 & Name(s) & Description of changes\\
... & ... & ...\\
\bottomrule
\end{tabularx}
\end{table}

~\newpage

\tableofcontents

~\newpage

\pagenumbering{arabic}

\wss{You are free to modify this template.}

\section{Introduction}

\wss{You can include your definition of what a hazard is here.}

\section{Scope and Purpose of Hazard Analysis}

\section{System Boundaries and Components}

\section{Critical Assumptions}

\wss{These assumptions that are made about the software or system.  You should
minimize the number of assumptions that remove potential hazards.  For instance,
you could assume a part will never fail, but it is generally better to include
this potential failure mode.}

\section{Failure Mode and Effect Analysis}

\begin{table}[H]
\caption{Failure Mode and Effect Analysis}
\begin{tblr}{
    |X[3,l]|X[3,l]|X[3,l]|X[3,l]|X[3,l]|X[r]|X[r]|
}
\hline
\hline
Design Function & 
Failure Modes  &  
Effects of Failure & 
Causes of Failure & 
Recommended Action & 
SR & 
Ref. \\
\hline
User authentication error & Invalid credentials & User cannot log in to system & User error or improperly saved data & Reset credentials and inform user & TODO & TODO \\
\hline
Database Access & Database is inaccessible & User cannot view progress or stored personal data & Database connection failure & Display static error page and await database backup/restoration & TODO & TODO \\
\hline
Working Application & Error state & User cannot view any pages, progress, and account & Major system failure due to bugs & Display static error page and await application restoration & TODO & TODO \\
\hline

 Hand sign tracking & Hand sign motions are too fast/slow & System cannot accurately give user feedback and measure progress & Model and system efficiency & Interrupt the user and inform user to adjust hand sign motions accordingly & TODO & TODO \\
\hline
Camera & Visual feed is unable to be captured & User's sign cannot be perceived by the device and subsequently be processed & 1. Poor Lighting Conditions & 1. Instruct user to adjust their environment lighting or move to environment with sufficient lighting & HR1 & H3-1 \\
  &  &  & 2. User skin color is not within the bounds of the training set data used & 2. Broaden training dataset to include more skin color types & DTR1 & H3-2 \\
  & Physical defect that impairs operation &  & 3. Cracked/Filthy camera lenses & 3. Notify user that camera lenses appear to be inoperable & HR2 & H4-1 \\
\hline
\end{tblr}
\label{table:nonlin} % is used to refer this table in the text
\end{table}

% \wss{Newly discovered requirements.  These should also be added to the SRS.  (A
% rationale design process how and why to fake it.)}
\section{Safety and Security Requirements}

\subsection{Hardware Requirements}

\subsection{Dataset Training Requirements}


\section{Roadmap}

% \wss{Which safety requirements will be implemented as part of the capstone timeline?
% Which requirements will be implemented in the future?}

\end{document}