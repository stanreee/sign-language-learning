\documentclass[12pt, titlepage]{article}

\usepackage{amsmath, mathtools}

\usepackage[round]{natbib}
\usepackage{amsfonts}
\usepackage{amssymb}
\usepackage{graphicx}
\usepackage{colortbl}
\usepackage{xr}
\usepackage{hyperref}
\usepackage{longtable}
\usepackage{xfrac}
\usepackage{tabularx}
\usepackage{float}
\usepackage{siunitx}
\usepackage{booktabs}
\usepackage{multirow}
\usepackage[section]{placeins}
\usepackage{caption}
\usepackage{fullpage}

\hypersetup{
bookmarks=true,     % show bookmarks bar?
colorlinks=true,       % false: boxed links; true: colored links
linkcolor=red,          % color of internal links (change box color with linkbordercolor)
citecolor=blue,      % color of links to bibliography
filecolor=magenta,  % color of file links
urlcolor=cyan          % color of external links
}

\usepackage{array}

\externaldocument{../../SRS/SRS}

\input{../../Comments}
%% Common Parts

\newcommand{\progname}{Software Engineering} % PUT YOUR PROGRAM NAME HERE
\newcommand{\authname}{Team 15, ASLingo
\\ Andrew Kil
\\ Cassidy Baldin
\\ Edward Zhuang
\\ Jeremy Langner
\\ Stanley Chan} % AUTHOR NAMES                  

\usepackage{hyperref}
    \hypersetup{colorlinks=true, linkcolor=blue, citecolor=blue, filecolor=blue,
                urlcolor=blue, unicode=false}
    \urlstyle{same}
                                


\begin{document}

\title{Module Interface Specification for \progname{}}

\author{\authname}

\date{\today}

\maketitle

\pagenumbering{roman}

\section{Revision History}

\begin{tabularx}{\textwidth}{p{3cm}p{2cm}X}
\toprule {\bf Date} & {\bf Version} & {\bf Notes}\\
\midrule
Jan 16, 2024 & 1.0 & Andrew, Stan, Edward; Finished back-end MIS breakdown\\
Date 2 & 1.1 & Notes\\
\bottomrule
\end{tabularx}

~\newpage

\section{Symbols, Abbreviations and Acronyms}

See SRS Documentation at \wss{give url}

\wss{Also add any additional symbols, abbreviations or acronyms}

\newpage

\tableofcontents

\newpage

\pagenumbering{arabic}

\section{Introduction}

The following document details the Module Interface Specifications for
\wss{Fill in your project name and description}

Complementary documents include the System Requirement Specifications
and Module Guide.  The full documentation and implementation can be
found at \url{...}.  \wss{provide the url for your repo}

\section{Notation}

\wss{You should describe your notation.  You can use what is below as
  a starting point.}

The structure of the MIS for modules comes from \citet{HoffmanAndStrooper1995},
with the addition that template modules have been adapted from
\cite{GhezziEtAl2003}.  The mathematical notation comes from Chapter 3 of
\citet{HoffmanAndStrooper1995}.  For instance, the symbol := is used for a
multiple assignment statement and conditional rules follow the form $(c_1
\Rightarrow r_1 | c_2 \Rightarrow r_2 | ... | c_n \Rightarrow r_n )$.

The following table summarizes the primitive data types used by \progname. 

\begin{center}
\renewcommand{\arraystretch}{1.2}
\noindent 
\begin{tabular}{l l p{7.5cm}} 
\toprule 
\textbf{Data Type} & \textbf{Notation} & \textbf{Description}\\ 
\midrule
character & char & a single symbol or digit\\
integer & $\mathbb{Z}$ & a number without a fractional component in (-$\infty$, $\infty$) \\
natural number & $\mathbb{N}$ & a number without a fractional component in [1, $\infty$) \\
real & $\mathbb{R}$ & any number in (-$\infty$, $\infty$)\\
\bottomrule
\end{tabular} 
\end{center}

\noindent
The specification of \progname \ uses some derived data types: sequences, strings, and
tuples. Sequences are lists filled with elements of the same data type. Strings
are sequences of characters. Tuples contain a list of values, potentially of
different types. In addition, \progname \ uses functions, which
are defined by the data types of their inputs and outputs. Local functions are
described by giving their type signature followed by their specification.

\section{Module Decomposition}

The following table is taken directly from the Module Guide document for this project.

\begin{table}[h!]
\centering
\begin{tabular}{p{0.3\textwidth} p{0.6\textwidth}}
\toprule
\textbf{Level 1} & \textbf{Level 2}\\
\midrule

\multirow{1}{0.3\textwidth}{Hardware-Hiding Module} 
& Video Input Module \mref{m8}\\
\midrule

\multirow{4}{0.3\textwidth}{Behaviour-Hiding Module} 
& Hand Sign Recognition Module \mref{m1}\\
& Controller Module \mref{m3}\\
& Data Processing Module \mref{m5}\\
& Machine Learning Module \mref{m6}\\
& Landing Page Module \mref{m9}\\
& Exercise Module \mref{m10}\\
& Login/Sign Up Module \mref{m12}\\
\midrule

\multirow{3}{0.3\textwidth}{Software Decision Module} 
& Hand Sign Verification Module \mref{m2}\\
& Data Collection Module \mref{m4}\\
& Testing and Verification Module \mref{m7}\\
& Exercise Selection/History Module \mref{m11}\\
& Account Management Module \mref{m13}\\
\bottomrule

\end{tabular}
\caption{Module Hierarchy}
\label{TblMH}
\end{table}

\newpage
~\newpage

% ---------------Template------------------
% \section{MIS of \wss{Module Name}} \label{Module} \wss{Use labels for
%   cross-referencing}

% \wss{You can reference SRS labels, such as R\ref{R_Inputs}.}

% \wss{It is also possible to use \LaTeX for hypperlinks to external documents.}

% \subsection{Module}

% \wss{Short name for the module}

% \subsection{Uses}


% \subsection{Syntax}

% \subsubsection{Exported Constants}

% \subsubsection{Exported Access Programs}

% \begin{center}
% \begin{tabular}{p{2cm} p{4cm} p{4cm} p{2cm}}
% \hline
% \textbf{Name} & \textbf{In} & \textbf{Out} & \textbf{Exceptions} \\
% \hline
% \wss{accessProg} & - & - & - \\
% \hline
% \end{tabular}
% \end{center}

% \subsection{Semantics}

% \subsubsection{State Variables}

% \wss{Not all modules will have state variables.  State variables give the module
%   a memory.}
% \subsubsection{Environment Variables}
% \wss{This section is not necessary for all modules.  Its purpose is to capture
%   when the module has external interaction with the environment, such as for a
%   device driver, screen interface, keyboard, file, etc.}
% \subsubsection{Assumptions}
% \wss{Try to minimize assumptions and anticipate programmer errors via
%   exceptions, but for practical purposes assumptions are sometimes appropriate.}
% \subsubsection{Access Routine Semantics}
% \noindent \wss{accessProg}():
% \begin{itemize}
% \item transition: \wss{if appropriate} 
% \item output: \wss{if appropriate} 
% \item exception: \wss{if appropriate} 
% \end{itemize}
% \wss{A module without environment variables or state variables is unlikely to
%   have a state transition.  In this case a state transition can only occur if
%   the module is changing the state of another module.}
% \wss{Modules rarely have both a transition and an output.  In most cases you
%   will have one or the other.}
% \subsubsection{Local Functions}
% \wss{As appropriate} \wss{These functions are for the purpose of specification.
%   They are not necessarily something that is going to be implemented
%   explicitly.  Even if they are implemented, they are not exported; they only
%   have local scope.}
% ---------------Template------------------

% --------------Hand Sign Recog------------------
\section{MIS of Hand Sign Recognition Module} \label{Module} 
\subsection{Module}
HSR
\subsection{Uses}
Machine Learning Module, Video Input Module
\subsection{Syntax}
% \subsubsection{Exported Constants}
\subsubsection{Exported Access Programs}
\begin{center}
\begin{tabular}{p{4cm} p{2cm} p{2cm} p{4cm}}
\hline
\textbf{Name} & \textbf{In} & \textbf{Out} & \textbf{Exceptions} \\
\hline
determine\_handsign & - & String & TIME\_LIMIT\_REACHED \\
\hline
\end{tabular}
\end{center}
\subsection{Semantics}
\subsubsection{State Variables}
\begin{itemize}
    \item MAX\_DECISION\_FRAMES - Frames needed to determine when the user has settled on a hand sign
    \item TIMEOUT\_LIMIT - Amount of time in seconds before the user automatically fails the question
\end{itemize}
\subsubsection{Environment Variables}
None
\subsubsection{Assumptions}
None
\subsubsection{Access Routine Semantics}
\noindent determine\_handsign():
\begin{itemize}
% \item transition: \wss{if appropriate} 
\item output: The name of the determined handsign
\item exception: exc := TIME\_LIMIT\_REACHED
\end{itemize}
\subsubsection{Local Functions}
process\_features()
% --------------Hand Sign Recog------------------

% --------------Hand Sign Verif------------------
\section{MIS of Hand Sign Verification Module} \label{Module} 
\subsection{Module}
HSV
\subsection{Uses}
Hand Sign Recognition Module, Controller
\subsection{Syntax}
% \subsubsection{Exported Constants}
\subsubsection{Exported Access Programs}
\begin{center}
\begin{tabular}{p{4cm} p{4cm} p{4cm} p{2cm}}
\hline
\textbf{Name} & \textbf{In} & \textbf{Out} & \textbf{Exceptions} \\
\hline
verify\_handsign & - & Boolean & - \\
\hline
\end{tabular}
\end{center}
\subsection{Semantics}
\subsubsection{State Variables}
None
\subsubsection{Environment Variables}
None
\subsubsection{Assumptions}
None
\subsubsection{Access Routine Semantics}
\noindent verify\_handsign():
\begin{itemize}
% \item transition: \wss{if appropriate} 
\item output: True/False for if the expected handsign was made 
\item exception: exc := None 
\end{itemize}
% \subsubsection{Local Functions}

% --------------Hand Sign Verif------------------

% ----------------Controller--------------------
\section{MIS of Controller Module} \label{Module} 
\subsection{Module}
Controller
\subsection{Uses}
Exercise Module, Hand Sign Verification Module
\subsection{Syntax}
% \subsubsection{Exported Constants}
\subsubsection{Exported Access Programs}
\begin{center}
\begin{tabular}{p{5cm} p{2cm} p{2cm} p{2cm}}
\hline
\textbf{Name} & \textbf{In} & \textbf{Out} & \textbf{Exceptions} \\
\hline
send\_requested\_handsign & String & None & - \\
get\_requested\_handsign & None & String & - \\
send\_passFail & Bool & None & - \\
get\_passFail & None & Bool & - \\
\hline
\end{tabular}
\end{center}
\subsection{Semantics}
\subsubsection{State Variables}
\begin{itemize}
    \item h - handsign variable to store intermediary data
    \item pass - Boolean to determine if the question was answered correctly
\end{itemize}
\subsubsection{Environment Variables}
None
\subsubsection{Assumptions}
None
\subsubsection{Access Routine Semantics}
\noindent send\_requested\_handsign():
\begin{itemize}
% \item transition: \wss{if appropriate} 
\item output: None 
\item exception: exc := None
\end{itemize}

\noindent get\_requested\_handsign():
\begin{itemize}
% \item transition: \wss{if appropriate} 
\item output: The expected handsign being asked by the front-end
\item exception: exc := None
\end{itemize}

\noindent send\_passFail():
\begin{itemize}
% \item transition: \wss{if appropriate} 
\item output: None
\item exception: exc := None
\end{itemize}

\noindent get\_passFail():
\begin{itemize}
% \item transition: \wss{if appropriate} 
\item output: The result of comparing the expected answer to what the back-end determined
\item exception: exc := None
\end{itemize}
% \subsubsection{Local Functions}
% ----------------Controller--------------------

% --------------Data Collection------------------
\section{MIS of Data Collection Module} \label{Module} 
\subsection{Module}
DCM
\subsection{Uses}
None
\subsection{Syntax}
% \subsubsection{Exported Constants}
\subsubsection{Exported Access Programs}
\begin{center}
\begin{tabular}{p{4cm} p{5cm} p{2cm} p{2cm}}
\hline
\textbf{Name} & \textbf{In} & \textbf{Out} & \textbf{Exceptions} \\
\hline
read\_training\_set & training\_imgs\_path & - & - \\
\hline
\end{tabular}
\end{center}
\subsection{Semantics}
\subsubsection{State Variables}
None
\subsubsection{Environment Variables}
None
\subsubsection{Assumptions}
None
\subsubsection{Access Routine Semantics}
\noindent read\_training\_set():
\begin{itemize}
\item transition: training.csv updated with raw training data
\item output: none
\item exception: exc := None 
\end{itemize}
% --------------Data Collection------------------

% --------------Data Processing------------------
\section{MIS of Data Processing Module} \label{Module} 
\subsection{Module}
DPM
\subsection{Uses}
Data Collection Module
\subsection{Syntax}
\subsubsection{Exported Constants}
\subsubsection{Exported Access Programs}
\begin{center}
\begin{tabular}{p{5cm} p{3cm} p{2cm} p{2cm}}
\hline
\textbf{Name} & \textbf{In} & \textbf{Out} & \textbf{Exceptions} \\
\hline
process\_training\_data & training.csv & - & - \\
\hline
\end{tabular}
\end{center}
\subsection{Semantics}
\subsubsection{State Variables}
None
\subsubsection{Environment Variables}
None
\subsubsection{Assumptions}
None
\subsubsection{Access Routine Semantics}
\noindent process\_training\_data():
\begin{itemize}
    \item transition: training.csv updated with processed training data
    \item output: none
    \item exception: exc := None
\end{itemize}
% --------------Data Processing------------------

% --------------Machine Learning------------------
\section{MIS of Machine Learning Module} \label{Module} 
\subsection{Module}
MLM
\subsection{Uses}
Data Processing Module
\subsection{Syntax}
\subsubsection{Exported Constants}
None
\subsubsection{Exported Access Programs}
\begin{center}
\begin{tabular}{p{4cm} p{5cm} p{2cm} p{2cm}}
\hline
\textbf{Name} & \textbf{In} & \textbf{Out} & \textbf{Exceptions} \\
\hline
SignLangModel & processed\_frame\_data & String & - \\
\hline
\end{tabular}
\end{center}
\subsection{Semantics}
\subsubsection{State Variables}
None
\subsubsection{Environment Variables}
None
\subsubsection{Assumptions}
None
\subsubsection{Access Routine Semantics}
SignLangModel():
\begin{itemize}
    \item transition: none
    \item output: Predicted hand sign for given processed frame data
    \item exception: none
\end{itemize}
\subsubsection{Local Functions}
train()
% --------------Machine Learning------------------

% --------------Test-Verif------------------
\section{MIS of Testing and Verification Module} \label{Module} 
\subsection{Module}
Tester
\subsection{Uses}
Hand Sign Verification Module
\subsection{Syntax}
% \subsubsection{Exported Constants}
% None
% \subsubsection{Exported Access Programs}
% \begin{center}
% \begin{tabular}{p{5cm} p{2cm} p{2cm} p{2cm}}
% \hline
% \textbf{Name} & \textbf{In} & \textbf{Out} & \textbf{Exceptions} \\
% \hline
% - & - & - & - \\
% \hline
% \end{tabular}
% \end{center}
\subsection{Semantics}
\subsubsection{State Variables}
None
\subsubsection{Environment Variables}
None
\subsubsection{Assumptions}
None
\subsubsection{Access Routine Semantics}
None
\subsubsection{Local Functions}
testDataCollectionModule()\\
testDataProcessingModule()\\
testMachineLearningModule()\\
testVideoInputModule()\\
testHandSignRecognitionModule()\\
testHandSignVerificationModule()\\
% --------------Test-Verif------------------

% --------------Video Input------------------
\section{MIS of Video Input Module} \label{Module} 
\subsection{Module}
Cam
\subsection{Uses}
None
\subsection{Syntax}
\subsubsection{Exported Access Programs}
\begin{center}
\begin{tabular}{p{4cm} p{3cm} p{4cm} p{2cm}}
\hline
\textbf{Name} & \textbf{In} & \textbf{Out} & \textbf{Exceptions} \\
\hline
get\_frame\_data & video input & NumPy ndarray & - \\
\hline
\end{tabular}
\end{center}
\subsection{Semantics}
\subsubsection{State Variables}
None
\subsubsection{Environment Variables}
None
\subsubsection{Assumptions}
None
\subsubsection{Access Routine Semantics}
get\_frame\_data():
\begin{itemize}
    \item transition: raw video input is turned into an array of shape (height, width, channels)
    \item output: the frame data read through the video feed in terms of a NumPy array
    \item exception: exc := None
\end{itemize}

% --------------Video Input------------------

\newpage

\bibliographystyle {plainnat}
\bibliography {../../../refs/References}

\newpage

\section{Appendix} \label{Appendix}

\wss{Extra information if required}

\end{document}