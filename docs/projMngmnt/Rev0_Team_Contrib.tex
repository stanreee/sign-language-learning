\documentclass{article}

\usepackage{float}
\restylefloat{table}

\usepackage{booktabs}

\title{Team Contributions: Rev 0\\\progname}

\author{\authname}

\date{}

%% Comments

\usepackage{color}

\newif\ifcomments\commentstrue %displays comments
%\newif\ifcomments\commentsfalse %so that comments do not display

\ifcomments
\newcommand{\authornote}[3]{\textcolor{#1}{[#3 ---#2]}}
\newcommand{\todo}[1]{\textcolor{red}{[TODO: #1]}}
\else
\newcommand{\authornote}[3]{}
\newcommand{\todo}[1]{}
\fi

\newcommand{\wss}[1]{\authornote{blue}{SS}{#1}} 
\newcommand{\plt}[1]{\authornote{magenta}{TPLT}{#1}} %For explanation of the template
\newcommand{\an}[1]{\authornote{cyan}{Author}{#1}}

%% Common Parts

\newcommand{\progname}{Software Engineering} % PUT YOUR PROGRAM NAME HERE
\newcommand{\authname}{Team 15, ASLingo
\\ Andrew Kil
\\ Cassidy Baldin
\\ Edward Zhuang
\\ Jeremy Langner
\\ Stanley Chan} % AUTHOR NAMES                  

\usepackage{hyperref}
    \hypersetup{colorlinks=true, linkcolor=blue, citecolor=blue, filecolor=blue,
                urlcolor=blue, unicode=false}
    \urlstyle{same}
                                


\begin{document}

\maketitle

\section{Demo Plans}

We plan on starting the demo with a quick introduction using sign language. We willl then walk through the front page UI, going through the 3 key pages:
\begin{enumerate}
    \item The hand sign resources page
    \item The hand sign practice page
    \item The exercise page
\end{enumerate}
Afterwards, we will leave time for questions with the Professor and TA.

\section{Meeting Attendance}

% \wss{For each team member how many team meetings have they attended since the
% POC demo.  This number should be determined from the meeting issues in the
% team's repo.  The first entry in the table should be the total number of team
% meetings held by the team.}

\begin{table}[H]
\centering
\begin{tabular}{ll}
\toprule
\textbf{Student} & \textbf{Meetings}\\
\midrule
Total & 22\\
Cassidy Bladin & 4\\
Stanley Chan & 6\\
Andrew Kil & 6\\
Jeremy Langner & 4\\
Edward Zhuang & 4\\
\bottomrule
\end{tabular}
\end{table}

\noindent Back-end Team held meetings on our own that Jeremy and Cassidy did not have to attend. Edward had work obligations that led him to miss a couple team meetings.

\section{Lecture Attendance}

% \wss{For each team member how many lectures have they attended since the POC
% demo.  This number should be determined from the lecture issues in the team's
% repo.  The first entry in the table should be the total number of lectures since
% the POC demo.}

\begin{table}[H]
\centering
\begin{tabular}{ll}
\toprule
\textbf{Student} & \textbf{Lectures}\\
\midrule
Total & 2\\
Cassidy Bladin & 2\\
Stanley Chan & 2\\
Andrew Kil & 2\\
Jeremy Langner & 2\\
Edward Zhuang & 2\\
\bottomrule
\end{tabular}
\end{table}

\section{Commits}

% \wss{For each team member how many commits to the main branch have been made
% since the POC demo.  The total is the total number of commits for the entire
% team since the POC demo.  The percentage is the percentage of the total commits
% made by each team member.}

\begin{table}[H]
\centering
\begin{tabular}{lll}
\toprule
\textbf{Student} & \textbf{Commits} & \textbf{Percent}\\
\midrule
Total & 126 & 100\% \\
Cassidy Bladin & 9 & 22\% \\
Stanley Chan & 3 & 21\% \\
Andrew Kil & 12 & 20\% \\
Jeremy Langner & 6 & 20\% \\
Edward Zhuang & 3 & 17\% \\
\bottomrule
\end{tabular}
\end{table}

\section{Issue Tracker}

% \wss{For each team member how many issues have they authored (including open and
% closed issues) and how many have they been assigned (only counting closed
% issues).}

\begin{table}[H]
\centering
\begin{tabular}{lll}
\toprule
\textbf{Student} & \textbf{Authored (O+C)} & \textbf{Assigned (C only)}\\
\midrule
Cassidy Bladin & 0 & 0 \\
Stanley Chan & 0 & 0 \\
Andrew Kil & 0 & 0 \\
Jeremy Langner & 0 & 0 \\
Edward Zhuang & 0 & 0 \\
\bottomrule
\end{tabular}
\end{table}

We unfortunately haven't making use of Git issue tracking as much as we should. Any issue has been communicated using our internal Discord and resolved through said Discord.

\section{CICD}

CICD is used in our project all the time. As stated in our Development plan, we open new branches for development features continuously committing to them until the feature is complete. Afterwards, we merge the branch to main and integrate the feature. 

\end{document}
