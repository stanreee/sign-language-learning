\documentclass[12pt, titlepage]{article}

\usepackage{booktabs}
\usepackage{tabularx}
\usepackage{hyperref}
\hypersetup{
    colorlinks,
    citecolor=black,
    filecolor=black,
    linkcolor=red,
    urlcolor=blue
}
\usepackage[round]{natbib}

\usepackage{amsmath, amsfonts}
\usepackage[margin=1in]{geometry}
\usepackage{amssymb}
\usepackage{amsthm}
\usepackage{amsmath}
\usepackage{multirow}
\usepackage{verbatim}
\usepackage{listings}
\usepackage{color}
\usepackage{hyperref}
\usepackage{blindtext}
\usepackage{cancel}
\usepackage{float}
\usepackage{enumitem}
\usepackage{graphicx}    

\usepackage[table]{xcolor}
\setlength{\tabcolsep}{18pt}
\renewcommand{\arraystretch}{1.5}
\renewcommand{\labelenumi}{\theenumi.}
\renewcommand{\labelenumii}{\theenumii.}
\renewcommand{\labelenumiii}{\theenumiii.}
\newcommand{\be}{\begin{enumerate}}
\newcommand{\ee}{\end{enumerate}}
\newcommand{\bi}{\begin{itemize}}
\newcommand{\ei}{\end{itemize}}
\newcommand{\bc}{\begin{center}}
\newcommand{\ec}{\end{center}}
\newcommand{\bv}{\begin{verbatim}}
\newcommand{\ev}{\end{verbatim}}
\newcommand{\ba}{\begin{align*}}
\newcommand{\ea}{\end{align*}}
\newcommand{\beq}{\begin{equation*}}
\newcommand{\eeq}{\end{equation*}}
\newcommand{\bs}{\begin{split}}
\newcommand{\es}{\end{split}}
\newcommand{\mname}[1]{\mbox{\sf #1}}
\newcommand{\pnote}[1]{{\langle \text{#1} \rangle}}
\renewcommand{\labelenumii}{\theenumii.}


\title{Software Requirements Specification\\ASLingo Application}

\author{Team 15, ASLingo
		\\ Andrew Kil
		\\ Cassidy Baldin
		\\ Edward Zhuang
		\\ Jeremy Langner
		\\ Stanley Chan
}

\date{\today}

%% Comments

\usepackage{color}

\newif\ifcomments\commentstrue %displays comments
%\newif\ifcomments\commentsfalse %so that comments do not display

\ifcomments
\newcommand{\authornote}[3]{\textcolor{#1}{[#3 ---#2]}}
\newcommand{\todo}[1]{\textcolor{red}{[TODO: #1]}}
\else
\newcommand{\authornote}[3]{}
\newcommand{\todo}[1]{}
\fi

\newcommand{\wss}[1]{\authornote{blue}{SS}{#1}} 
\newcommand{\plt}[1]{\authornote{magenta}{TPLT}{#1}} %For explanation of the template
\newcommand{\an}[1]{\authornote{cyan}{Author}{#1}}

%% Common Parts

\newcommand{\progname}{Software Engineering} % PUT YOUR PROGRAM NAME HERE
\newcommand{\authname}{Team 15, ASLingo
\\ Andrew Kil
\\ Cassidy Baldin
\\ Edward Zhuang
\\ Jeremy Langner
\\ Stanley Chan} % AUTHOR NAMES                  

\usepackage{hyperref}
    \hypersetup{colorlinks=true, linkcolor=blue, citecolor=blue, filecolor=blue,
                urlcolor=blue, unicode=false}
    \urlstyle{same}
                                


\begin{document}

\maketitle

\pagenumbering{roman}
\tableofcontents
\listoftables
\listoffigures

\begin{table}[bp]
\caption{Revision History}
\begin{tabularx}{\textwidth}{|l|l|X|}
\toprule {\bf Date} & {\bf Developers} & {\bf Change}\\
\midrule
September 25, 2023 & All team members & Initial draft, added some functional requirements \\
September 26, 2023 & Andrew Kil & Added constraints and naming conventions \\
September 26, 2023 & Cassidy Baldin & Added some functional and non-functional requirements\\
Date & Name & Change\\
\bottomrule
\end{tabularx}
\end{table}

\newpage

\pagenumbering{arabic}

This document describes the requirements for ASLingo. The template for the Software
Requirements Specification (SRS) is a subset of the Volere
template \textit{Robertson And Robertson (2012)}.  Subsections \textit{Clients} and \textit{Customers} were removed due to not having any such dependents.

\section{Project Drivers}

\subsection{The Purpose of the Project}

Learning a new language can be an arduous task that only gets more challenging
with age, as individuals may find it difficult to dedicate time and effort to
it. American Sign Language (ASL) is particularly hard due to its visual and
gestural nature, which is not found in other, verbal languages. The purpose of this project is
to ease that challenge by providing an online, easy-to-access web platform for
individuals to learn new signs and test their comprehension at their own pace
in a fun, interactive manner. Focusing in on consistent effort and continuous
feedback, ASLingo provides real-time guidance to ensure users stay on track to
achieving their goals of learning ASL.

\subsection{The Stakeholders}

 The stakeholders for this project include those who use sign language as their primary mode of communication in daily life as well as those who have an interest in learning ASL. This would naturally expand outward towards educators who wish to promote the learning of ASL to their respective institutions. 

% \subsubsection{The Client}

% \subsubsection{The Customers}

\subsubsection{Other Stakeholders}

\subsection{Mandated Constraints}

The project is constrained by the following:

\begin{itemize}
    \item The Project Expenses Cannot Exceed \$750
\end{itemize}

\subsection{Naming Conventions and Terminology}

\begin{table}
\caption{Naming Conventions and Terminology}
\noindent \begin{tabularx}{\textwidth}{|p{0.3\linewidth}|X|}
\toprule
\textbf{Term, Abbreviation, or Acronym} & \textbf{Description}\\
\midrule
A
& Shorthand for Assumption\\
\hline
ASL
& Shorthand for American Sign Language. It is a form of sign language primarily used in the US and in parts of Canada\\
\hline
ASLingo
& The commerical name for the project\\
\bottomrule
\end{tabularx}
\end{table}

\subsection{Relevant Facts and Assumptions}

User characteristics should go under assumptions.

\section{Functional Requirements}

\subsection{The Scope of the Work and the Product}

\subsubsection{The Context of the Work}

\subsubsection{Work Partitioning}

\subsubsection{Individual Product Use Cases}

\subsection{Functional Requirements}
 
** see Table 3: Functional Requirements of ASLingo, might need to format this differently \\

\begin{table}
\caption{Functional Requirements of ASLingo}
\noindent \begin{tabular}{| c | p{4cm}| p{5cm}|}
\toprule 
\textbf{Requirement No.} & \textbf{Description} &\textbf{Rationale}\\
\midrule
FR1 & The system should be able to connect with a camera. & \\
\hline
FR2 & The system should be able to recognize American Sign Language hand signs. & \\
\hline
FR3 & The system should allow users to create an account. & \\
\hline
FR4 & The system should allow users to sign into their account if it exists. & \\
\hline
FR5 & The system should provide a diagnostic quiz for new users. & \\
\hline
FR6 & The system should provide a progression based course for ASL. & \\
\hline
FR7 & The system should save user progress. & \\
\hline
FR8 & The system should allow users to access the program via a web application (functional or a constraint?) & \\
\hline
FR9 & The system should be able to communicate to the user if they have answered the prompt correctly. & \\ % I think the phrasing for this is fine
\hline
FR10 & The system should notify the user of any potential errors that may arise during camera recognition. & \\
\bottomrule
\end{tabular}
\end{table}

\section{Non-functional Requirements}

\subsection{Look and Feel Requirements}

The system should remind users of similar language learning applications (familiarity with the system, fit criterion is a sample of users could rate how familiar/easy to use it is compared to other language learning apps) \\

The system should show the user how much progress they have made in their learning (level system, progress bar, score after a quiz etc.) \\

The system should clearly show the user if they have answered the prompt correctly. \\

\subsection{Usability and Humanity Requirements}

The system should be able to be used by people with little to no training (intro tutorial maybe but should be easy to understand; ease of use) \\

The system should be able to be used by people who are hard of hearing or deaf, as well as those who are able to hear (accessibility) \\

The system should allow users to personalize their account (avatar, name, progress; personalization) \\

\subsection{Performance Requirements}

*can change the time/percents shown* \\ 

The system should respond to user input within 1? second (speed/latency) \\

The system should be able to accurately determine the sign shown by the user 90\%? of the time (accuracy) \\ 

The system should be able to accurately match the user input sign to the prompt given by the system 90\%? of the time (accuracy) \\

The system should be able to host ??? users at one time (capacity) \\

The system should allow for new signs to be added over the lifespan of the system (scalability/longevity) \\

The system should show the user if the input needs to be adjusted (fault-tolerance? functional?) \\

\subsection{Operational and Environmental Requirements}

The system should be used as a web application on a browser/laptop (expected physical environment) \\

The system should be able to access a user's camera device \\

\subsection{Maintainability and Support Requirements}

The system should be tested regularly to ensure it's functionality and usability (maintenance) \\

%% Maybe the scalability/longevity requirement from performance?

\subsection{Security Requirements}

The system should allow the user to access their account after creating it (access) \\

The system should ensure that incorrect input to the system is used (integrity) \\

The system should store user account info securely? or keep user account info private? (privacy) \\

\subsection{Cultural Requirements}

The system should be written in Canadian English and teach users using American Sign Language. \\

\subsection{Legal Requirements}

The system should adhere to user privacy laws?  \\

The system should not train the model on personal/confidential/illegal data? \\


\subsection{Health and Safety Requirements}

This section is not in the original Volere template, but health and safety are
issues that should be considered for every engineering project.

\section{Project Issues}

\subsection{Open Issues}
There are currently no open issues with the project at the moment.
\subsection{Off-the-Shelf Solutions}
\begin{enumerate}
    \item ASL App is a mobile exclusive platform with 2500+ signs and phrases to teach ASL via short video clips. This app offers 4 packs for free with the basics like the alphabet, numbers, and universal gestures. Paid packs are also available ex. compliments, moods, and social gestures for \$0.99. The app ultimately serves as a mobile hub for common expressions to study and learn wherever you are.
    \item Canadian Hearing Services offer both in-person and virtual educational ASL courses for a variety of experience levels. These courses educate via teacher instruction, role play, videos, and work books.
    A variety of other public, private, and educational institutes offer similar courses and instructional content.
\end{enumerate}

\subsection{New Problems}

N/A

\subsection{Tasks}
NOT SURE?
\subsection{Migration to the New Product}
N/A
\subsection{Risks}

\begin{enumerate}
    \item The primary risk of this product is the potential for error when trying to analyze and recognize a user's sign to give feedback or determine if their form is correct. This could cause users to improperly learn signs and hinder their learning.
\end{enumerate}

\subsection{Costs}
\begin{enumerate}
    \item Website domain -TBA
    \item App hosting platform - TBA
    \item Database -TBA
\end{enumerate}
\subsection{User Documentation and Training}

The system and it's interface design should be intuitive enough to learn how to use the app.

SHOULD PROBABLY ADD SOMETHING ELSE!

\subsection{Waiting Room}

N/A

\subsection{Ideas for Solutions}

N/A

\bibliographystyle{plainnat}

\bibliography{SRS}

\newpage

\section{Appendix}

This section has been added to the Volere template.  This is where you can place
additional information.

\subsection{Symbolic Parameters}

The definition of the requirements will likely call for SYMBOLIC\_CONSTANTS.
Their values are defined in this section for easy maintenance.

\section{Appendix --- Reflection}

The information in this section will be used to evaluate the team members on the
graduate attribute of Lifelong Learning.  Please answer the following questions:

\begin{enumerate}
  \item Which of the courses you have taken, or are currently taking, will help
  your team to be successful with your capstone project.
  \item What knowledge and skills will the team collectively need to acquire to
  successfully complete this capstone project?  Examples of possible knowledge
  to acquire include domain specific knowledge from the domain of your
  application, or software engineering knowledge, mechatronics knowledge or
  computer science knowledge.  Skills may be related to technology, or writing,
  or presentation, or team management, etc.  You should look to identify at
  least one item for each team member.
  \item For each of the knowledge areas and skills identified in the previous
  question, what are at least two approaches to acquiring the knowledge or
  mastering the skill?  Of the identified approaches, which will each team
  member pursue, and why did they make this choice?
\end{enumerate}


\end{document}